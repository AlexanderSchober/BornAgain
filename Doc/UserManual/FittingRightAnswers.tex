\section {How to get the right answer from fitting} 
\SecLabel{FittingRightAnswers}

%As it has already been mentioned in \SecRef{FittingGentleIntroducion}, 
One of the main difficulties in fitting the data with the model 
is the presence of multiple
local minima in the objective function. Many problems can cause the
fit to fail, for example:
\begin{itemize}
\item an unreliable physical model,
\item an unappropriate choice of objective function
\item multiple local minima,
\item an unphysical behavior of the objective function, unphysical regions
  in the parameters space,
\item an unreliable parameter error calculation in the presence of
  limits on the parameter value,
\item an exponential behavior of the objective function and the
  corresponding numerical inaccuracies, excessive numerical roundoff
  in the calculation of its value and derivatives,
\item large correlations between parameters,
\item very different scales of parameters involved in the calculation,
\item not positive definite error matrix even at minimum.
\end{itemize}


The given list, of course, is not only related to \BornAgain\
fitting. It remains applicable to any fitting program and any kind of theoretical model.
%To address all these difficulties some amount of manual tuning might be necessary.
 Below we give some recommendations which might help the user to achieve reliable fit results.

\subsection*{General recommendations}
\begin{itemize}
\item initially choose  a small number of free fitting parameters,
\item eliminate redundant parameters,
\item provide a good initial guess for the fit parameters,
\item start from the default minimizer settings and perform some fine tuning after some experience has been acquired,
\item repeat the fit using different starting values for the parameters or their limits,
\item repeat the fit, fixing and varying different groups of parameters,
%\item use \Code{Minuit2} minimizer with \Code{Migrad} algorithm
%  (default) to get the most reliable parameter error estimation,
%\item try \Code{GSLMultiFit} minimizer or \Code{Minuit2} minimizer with \Code{Fumili} %algorithm to get fewer iterations.


%\subsection*{Interpretation of errors.}


{\bf to be continued... }


\end{itemize}
