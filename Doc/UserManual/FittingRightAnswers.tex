\section {How to get right answer from fitting.} 
\SecLabel{FittingRightAnswers}

As it was already mentioned in \SecRef{FittingGentleIntroducion}, 
one of the main difficulties in fitting the data with the model 
is the presence of multiple
local minima in the objective function. The extended list of problems causing fit to failure includes
\begin{itemize}
\item unreliable physical model
\item multiple local minima
\item unphysical behavior of objective function, unphysical regions in parameter space
\item unreliable parameter error calculation in the presence of limits on parameter value
\item often exponential behavior of objective function and corresponding numerical inaccuracies and excessive numerical roundoff in calculation of its value and derivatives
\item large correlations between parameters
\item very different scale of parameters involved in calculation
\item not positive definite error matrix even at minimum
\end{itemize}


Given list, of course, is unrelated only to \BornAgain\ fitting. It remains the same
while fitting the data with any fitting program and any kind of theoretical model.
To address all these difficulties some amount of manual tuning might be necessary. Below we give some recommendations which might help the user to achieve reliable fit results.

\subsection*{General recommendation}
\begin{itemize}
\item initially choose small number of free fitting parameters
\item eliminate redundand parameters
\item provide a good initial guess for fit parameters
\item start from default minimizer settings and turn to the fine tuning after some experience has been acquired.
\item repeat fit using different starting values for parameters or their limits
\item repeat fit fixing and releasing different groups  of parameters
\item use \Code{Minuit2} minimizer with \Code{Migrad} algorithm (default) to get most reliable parameter error estimation
\item try \Code{GSLMultiFit} minimizer or \Code{Minuit2} minimizer with \Code{Fumili} algorithm to get fewer iterations


%\subsection*{Interpretation of errors.}


{\bf to be continued... }


\end{itemize}
