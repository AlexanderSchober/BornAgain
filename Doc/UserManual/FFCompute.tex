%%%%%%%%%%%%%%%%%%%%%%%%%%%%%%%%%%%%%%%%%%%%%%%%%%%%%%%%%%%%%%%%%%%%%%%%%%%%%%%%
%%
%%   BornAgain User Manual
%%
%%   homepage:   http://www.bornagainproject.org
%%
%%   copyright:  Forschungszentrum Jülich GmbH 2015
%%
%%   license:    Creative Commons CC-BY-SA
%%   
%%   authors:    Scientific Computing Group at MLZ Garching
%%               C. Durniak, M. Ganeva, G. Pospelov, W. Van Herck, J. Wuttke
%%
%%%%%%%%%%%%%%%%%%%%%%%%%%%%%%%%%%%%%%%%%%%%%%%%%%%%%%%%%%%%%%%%%%%%%%%%%%%%%%%%

\chapter{Shape transform computations}  \label{SFFcomp}

\index{Shape transform!computation|(}
\index{Form factor!computation|(}

%%%%%%%%%%%%%%%%%%%%%%%%%%%%%%%%%%%%%%%%%%%%%%%%%%%%%%%%%%%%%%%%%%%%%%%%%%%%%%%%
\section{Introduction}
%%%%%%%%%%%%%%%%%%%%%%%%%%%%%%%%%%%%%%%%%%%%%%%%%%%%%%%%%%%%%%%%%%%%%%%%%%%%%%%%

The Form factor of a hard-shell particle is simply its \E{shape transform},
as introduced in~(\ref{EFFhardshell}),
\index{Shape transform}%
\begin{equation}\label{EFF_SFF2}
  F(\q)=\int_V {\rm d}^3r\, {\rm e}^{i\q\r},
\end{equation}
where $V$ delimits the volume of the particle.
Here we show for a few selected cases
how to bring the analytical solution of this integral into a form
 suitable for speedy and stable computation.
This background material probably is of interest only for
developers who want to add some more form factors to \BornAgain's collection
(catalogued in App.~\ref{SFF}).

\index{Form factor!bound for absolute value}%
From~(\ref{EFF_SFF2}), it is immediately clear that~$F(\q)$ is bounded by
\begin{equation}
  |F(\q)|\le V,
\end{equation}
with equality holding for~$\q=0$.
This excludes the presence of nonremovable singularities.
However, analytical solutions of~(\ref{EFF_SFF2}) do contain
removable singularities.
These singularities must be investigated closely
in order to prevent floating-point errors and ill conditioned computations.

\index{Form factor!singularities}%
It seems that \E{all} analytical form factors have a singularity at~$\q=0$.
Furthermore, if the particle has \E{plane} surfaces,
then there is a singularity for all $\q$ that lie in one of these planes.
A simple example is provided by the cuboid form factor (Sect.~\ref{SBox})
\begin{equation}
F= L W H\exp\left(i q_z \frac{H}{2}\right) \sinc\left(q_x \frac{L}{2}\right)
\sinc\left(q_y \frac{W}{2}\right) \sinc\left(q_z \frac{H}{2}\right).
\end{equation}
If the \E{sinus cardinalis} is written as
\begin{equation}
  \sinc(u) = \frac{\sin(u)}{u},
\end{equation}
then there appears to be a singularity at $u=0$.
Applying L'Hôpital's rule, it is obvious
that this singularity can be removed by the interpolated value
$\sinc(0)=1$,
which is of course part of the definition of the special function~$\sinc$.

While all this is clear and easy,
computations become more involved
if a particle has planes that do not coincide with
the $xy$, $xz$, or~$yz$ planes of the Cartesian coordinate system.
Typically then, the straightforward analytical solution of~(\ref{EFF_SFF2})
results in messy expressions that obscure the symmetries of~$F(\q)$,
and make it difficult to recognize the singularities and
to remove them.
Therefore we consider a few such cases more closely.

%%%%%%%%%%%%%%%%%%%%%%%%%%%%%%%%%%%%%%%%%%%%%%%%%%%%%%%%%%%%%%%%%%%%%%%%%%%%%%%%
\section{Regular polyhedron}
%%%%%%%%%%%%%%%%%%%%%%%%%%%%%%%%%%%%%%%%%%%%%%%%%%%%%%%%%%%%%%%%%%%%%%%%%%%%%%%%

\index{Shape transform!computation|)}
\index{Form factor!computation|)}
