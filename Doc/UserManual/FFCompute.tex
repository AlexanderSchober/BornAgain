%%%%%%%%%%%%%%%%%%%%%%%%%%%%%%%%%%%%%%%%%%%%%%%%%%%%%%%%%%%%%%%%%%%%%%%%%%%%%%%%
%%
%%   BornAgain User Manual
%%
%%   homepage:   http://www.bornagainproject.org
%%
%%   copyright:  Forschungszentrum Jülich GmbH 2015
%%
%%   license:    Creative Commons CC-BY-SA
%%
%%   authors:    Scientific Computing Group at MLZ Garching
%%               C. Durniak, M. Ganeva, G. Pospelov, W. Van Herck, J. Wuttke
%%
%%%%%%%%%%%%%%%%%%%%%%%%%%%%%%%%%%%%%%%%%%%%%%%%%%%%%%%%%%%%%%%%%%%%%%%%%%%%%%%%

\chapter{Shape transform computations}  \label{SFFcomp}

\index{Shape transform!computation|(}
\index{Form factor!computation|(}

\def\FFP{F_\parallel}
\def\FFPm{\Delta_\parallel}
\def\expm{\text{exp}_{-1}}
\def\R{\v{R}}
\let\textE=\E
\def\E{\v{E}}
\def\Gp{\Gamma_\parallel}
\def\x{\v{x}}
\def\V{\v{V}}
\def\qp{\v{p}}
\def\n{\v{\hat n}}
\def\uq{\v{\hat q}}
\def\uqp{\v{\hat p}}

%%%%%%%%%%%%%%%%%%%%%%%%%%%%%%%%%%%%%%%%%%%%%%%%%%%%%%%%%%%%%%%%%%%%%%%%%%%%%%%%
\section{It's all about removable singularities}\label{SShapeTrafIntro}
%%%%%%%%%%%%%%%%%%%%%%%%%%%%%%%%%%%%%%%%%%%%%%%%%%%%%%%%%%%%%%%%%%%%%%%%%%%%%%%%

The Form factor of a hard-shell particle is simply its \E{shape transform},
as introduced in~(\ref{EFFhardshell}),
\begin{equation}\label{Eff3d}
  F(\q)=\int_V\!\d^3r\, \e^{i\q\r},
\end{equation}
where $V$ is the volume of the particle.
Here we show for a few selected cases
how to bring the analytical solution of this integral into a form
 suitable for speedy and stable computation.
This background material is of interest mostly for
developers who want to add some more form factors to \BornAgain's collection
(catalogued in App.~\ref{SFF}).

\index{Form factor!maximal absolute value}%
From~(\ref{Eff3d}), it is immediately clear that~$F(\q)$ is bounded by
\begin{equation}
  |F(\q)|\le V,
\end{equation}
with equality holding for~$\q=0$.
This excludes the presence of nonremovable singularities.
However, analytical solutions of~(\ref{Eff3d}) do contain
removable singularities.
These singularities must be investigated closely
in order to prevent floating-point errors and ill conditioned computations.

\index{Form factor!singularities}%
It seems that \E{all} analytical form factors have a singularity at~$\q=0$.
Furthermore, if the particle has \E{plane} surfaces,
then there is a singularity for all $\q$ that are normal to one of these planes.
A simple example is provided by the cuboid form factor (Sec.~\ref{SBox})
\begin{equation}\label{Eboxff}
F= L W H\exp\left(i q_z \frac{H}{2}\right) \sinc\left(q_x \frac{L}{2}\right)
\sinc\left(q_y \frac{W}{2}\right) \sinc\left(q_z \frac{H}{2}\right).
\end{equation}
If the \E{sinus cardinalis} is written as
\index{Sinus cardinalis@Sinus cardinalis (sinc)!removable singularity}
\begin{equation}\label{Esinc}
  \sinc(u) = \frac{\sin(u)}{u},
\end{equation}
then there appears to be a singularity at $u=0$.
Applying L'H\^opital's rule, it is obvious
that this singularity can be removed by the interpolated value
$\sinc(0)=1$,
as is done in the generally agreed definition of the special function~$\sinc$.

Unfortunately, other form factors are not as simple as~(\ref{Eboxff}).
Computations become more involved
if a particle has planes with normals that are not oriented along
any of the axes of the Cartesian coordinate system.
Typically then, the straightforward analytical solution of~(\ref{Eff3d})
results in messy expressions that obscure the symmetries of~$F(\q)$,
and make it difficult handle the singularities.
Instead of addressing particular geometric shapes,
we consider arbitrary polygons and polyhedra.
It turns out that these generic solutions
are easier to implement
than lengthy arithmetic expressions for specific geometries.

% TODO? another concern: integrals without numeric solution

%%%%%%%%%%%%%%%%%%%%%%%%%%%%%%%%%%%%%%%%%%%%%%%%%%%%%%%%%%%%%%%%%%%%%%%%%%%%%%%%
\section{Polygon}\label{SFFPolygon}
%%%%%%%%%%%%%%%%%%%%%%%%%%%%%%%%%%%%%%%%%%%%%%%%%%%%%%%%%%%%%%%%%%%%%%%%%%%%%%%%

\index{Form factor!polygon|(}%
\index{Polygon!form factor|(}

To derive the form factor of prisms and pyramids with polygonal bases
we will need the two-dimensional form factor
%\nomenclature[2f134 2q040]{$\FFP(\q,\Gamma)$}{Two-dimensional form factor of planar shape~$\Gamma$}%
\begin{equation}\label{Eff2d}
  \FFP(\q,\Gamma)
  \coloneqq \int_{\Gamma}\!\d^2r\, \e^{i\q\r}
\end{equation}
of an arbitrary planar polygon~$\Gamma$,
located somewhere in affine three-dimensional Euclidean space.
The solution available from the literature~\cite{LeMi83}
contains singularities and does not provide an obvious way how handle them.
Here, a different expression shall be derived by a different method.

The normal vector~$\n$ of~$\Gamma$ induces a decomposition
\begin{equation}
  \q_\perp\coloneqq(\n\q)\n,\text{~and~}\q_\parallel\coloneqq\q-\q_\perp,
\end{equation}
and similarly for~$\r$.
The integration variable in~(\ref{Eff2d}) influences only~$\r_\parallel$,
whereas~$\r_\perp$ is constant for all $\r\in\Gamma$.
The corresponding translation in affine space
shall be denoted as $\Gamma=\r_\perp+\Gamma_\parallel$.
This allows us to draw the $\r_\perp$~dependence in front of the integral,
\begin{equation}\label{EF2F2}
  \FFP(\q,\Gamma)
  = \e^{i\q_\perp\r_\perp}\,\FFP(\q_\parallel,\Gp).
\end{equation}
To keep notation light,
in the following we shall substitute $\qp$ for the dummy variable~$\q_\parallel$.
With the help of Stokes's theorem
\begin{equation}\label{EStokes}
  \iint\!\d r^2\,\n\cdot\Nabla\times\v{G} = \oint \d\v{r}\cdot\v{G},
\end{equation}
and with the choice $\v{G}\coloneqq\v{a}\, (\e^{i\qp\r}-1)$,
we find
\begin{equation}
  \FFP(\qp,\Gp)=\frac{\v{a}}{i\, \n(\qp\times\v{a})} \oint_{\partial\Gp} \d\v{r}\,(\e^{i\qp\r}-1).
\end{equation}
We define the absolute value
$p\coloneqq|\v{p}|=(\v{p}\v{p}^*)^{1/2}$,
and the unit vector $\v{\hat p}\coloneqq\v{p}/p$.
With the choice $\v{a}\coloneqq\n\times\qp^*$,
the form factor becomes
\begin{equation}\label{Elastoint}
  \FFP(\qp,\Gp)=\frac{\n\times\uqp^*}{ip} \oint_{\partial\Gp} \d\v{r}\,(\e^{i\qp\r}-1).
\end{equation}
Let the polygon have $N$ vertices $\V_0,\ldots,\V_{N-1}$,
and put $\V_N\coloneqq\V_0$.
The edges of the polygon shall parametrized by
\begin{equation}
  \r_j(\lambda) = \frac{\V_{j+1}+\V_j}{2} + \frac{\V_{j+1}-\V_j}{2} \lambda
  \eqqcolon \R_j + \E_j\lambda.
\end{equation}
with $-1\le\lambda\le+1$.
The line integral~(\ref{Elastoint}) then takes the form
\begin{equation}
    \FFP(\qp,\Gp)
   = \frac{\n\times\uqp^*}{ip} \sum_{j=0}^{N-1}
              \int_{-1}^{+1}\!\d\lambda\, \frac{\d\r_j}{\d\lambda}\,(\e^{i\qp\r}-1),
\end{equation}
and yields
\begin{equation}\label{Effpolygon1}
    \FFP(\qp,\Gp)
    = \frac{\n\times\uqp^*}{ip}
      \sum_{j=0}^{N-1} \E_j
      \left( \frac{\e^{i\qp(\R_j+\E_j)}-\e^{i\qp(\R_j-\E_j)}}{i\qp\E_j} - 2 \right).
\end{equation}
The removable singularity for $\qp\E_j=0$
can be handled by a sinc function, as discussed in Sec.~\ref{SShapeTrafIntro}.
\Emph{
\begin{equation}\label{Effpolygon3}
  \FFP(\qp,\Gp)
  = 2\,\n\times\uqp^*
    \sum_{j=0}^{N-1} \E_j \frac{\sinc(\qp\E_j) \e^{i\qp\R_j} - 1}{ip}.
\end{equation}
}
This expression is used in our implementation of \texttt{FormFactorPrism3}
(Sec.~\ref{SPrism3}).
Near the removable singularity at~$p=0$,
the scalar function of $p$, $\qp\E_j$ and $\qp\R_j$,
written as a fraction,
 is not computed term by term,
but from its series expansion.

If a polygon~$\Gamma_2$ has a two-fold symmetry axis,
then the form factor can be further simplified.
We write $N\eqqcolon2n$
and make use of $\E_{j+n}=-\E_j$ to transform~(\ref{Effpolygon3}) into
\Emph{
\begin{equation}\label{Eff2ngon}
    \FFP(\qp,\Gamma_{2\parallel}) = \displaystyle 4\, \n\times\uqp^*\sum_{j=0}^{n-1}
              \E_j (\uqp\R_j) \sinc(\qp\E_j) \sinc(\qp\R_j),
\end{equation}
}
where the singularity at $p=0$ is absorbed in a second sinc function.
This result is used in our implementation of \texttt{FormFactorPrism6}
(Sec.~\ref{SPrism6}).
\index{FormFactorPrism6@\Code{FormFactorPrism6}}%

From the definition~(\ref{Eff2d}) it is immediately clear
that $\FFP(0,\Gamma)$ is the \textE{area} of polygon~$\Gamma$.
To confirm this from our result~(\ref{Effpolygon3}),
we expand the numerator in~$\qp$ and retain the leading nonvanishing term.
We find
\begin{equation}\label{Effarea}
  \begin{array}{@{}lcl}
  \FFP(\qp\!\to\!0,\Gp)
    &\doteq& \displaystyle 2\sum \n(\uqp^*\times\E_j) (\uqp\R_j)\\[1.8ex]
    &=&\displaystyle \frac{1}{2}\sum \n\left\{
        (\uqp^*\times\V_{j+1})(\uqp\V_j)
      - (\uqp^*\times\V_j)(\uqp\V_{j+1})
      \right\}\\[1.8ex]
    &=&\displaystyle \frac{1}{2}\sum \n(\V_j\times\V_{j+1}),
  \end{array}
\end{equation}
which is the area of the polygon
as given by a tesselation by triangles with vertices at $\v{0},\V_j,\V_{j+1}$.


\index{Form factor!polygon|)}%
\index{Polygon!form factor|)}%


%%%%%%%%%%%%%%%%%%%%%%%%%%%%%%%%%%%%%%%%%%%%%%%%%%%%%%%%%%%%%%%%%%%%%%%%%%%%%%%%
\section{Prism}\label{SFFPrism}
%%%%%%%%%%%%%%%%%%%%%%%%%%%%%%%%%%%%%%%%%%%%%%%%%%%%%%%%%%%%%%%%%%%%%%%%%%%%%%%%

\index{Form factor!prism|(}%
\index{Prism (form factor)!generic polygonal|(}%

For a prism~$\Pi$ with polygonal base~$\Gamma$,
extending in $r_\perp$ from~0 up to a height~$H$,
the form factor is
\begin{equation}
  F(\q,\Pi)
  = \int_0^H\!\d r_\perp\,\e^{i q_\perp r_\perp}\,\FFP(\q_\parallel,\Gp)
\end{equation}
with the two-dimensional form factor $\FFP(\q_\parallel,\Gp)$ as determined in the preceding section.
The solution shall be written as
\Emph{%
\begin{equation}
  F(\q,\Pi)
  = H\, \expm(iq_\perp H)\,\FFP(\q_\parallel,\Gp)
\end{equation}}
with the function
\begin{equation}
  \expm(z) \coloneqq
  \left\{ \begin{array}{ll}
    \displaystyle \frac{\exp(z)-1}{z}&\text{~for~}z\ne0,\\[1.8ex]
    1&\text{~for~}z=0.
    \end{array}\right.
\end{equation}
To avoid cancellation in the numerator,
function values in a neighborhood of the removable singularity~$z=0$
are computed not term by term,
but from the series expansion
\begin{equation}
  \expm(z) = 1 + \frac{z}{2} + \ldots
\end{equation}

\index{Form factor!prism|)}%
\index{Prism (form factor)!generic polygonal|)}%

%%%%%%%%%%%%%%%%%%%%%%%%%%%%%%%%%%%%%%%%%%%%%%%%%%%%%%%%%%%%%%%%%%%%%%%%%%%%%%%%
\section{Polyhedron}\label{SFFPolyhedron}
%%%%%%%%%%%%%%%%%%%%%%%%%%%%%%%%%%%%%%%%%%%%%%%%%%%%%%%%%%%%%%%%%%%%%%%%%%%%%%%%

To derive the form factor of a polyhedron~$\Pi$,
we start from the divergence theorem
\begin{equation}
  \iiint_\Pi\!\d^3r\,\Nabla\v{H} = \iint_{\partial\Pi}\!\d^2r\, \n\v{H}.
\end{equation}
With the choice $\v{H}\coloneqq\q^*(\e^{i\q\r}-1)$,
we obtain
\begin{equation}\label{Eff3d1}
  F(\q,\Pi) = \frac{\q^*}{iq^2} \iint_{\partial\Pi}\!\d^2r\, \n \left(\e^{i\q\r}-1\right).
\end{equation}
The surface of a polyhedron consists of polygonal planar faces,
\begin{equation}
  \partial\Pi = \bigcup_k \Gamma_k.
\end{equation}
Each polygonal face~$\Gamma_k$ has a normal vector~$\n_k$,
and a two-dimensional form factor~$\FFP(q,\Gamma_k)$ as determined in Sec.~\ref{SFFPolygon}.
Using these, (\ref{Eff3d1}) can be written as
\Emph{
\begin{equation}\label{Eff3d2}
  F(\q,\Pi) = \uq^* \sum_k \n_k \FFPm(\q,\Gamma_k)
\end{equation}
}
with a function defined by
\begin{equation}
  \FFPm(\q,\Gamma)
  \coloneqq \frac{\FFP(\q,\Gamma)-\FFP(0,\Gamma)}{iq}
\end{equation}
except for $q=0$ where the analytic continuation shall be
used to remove the apparent singularity.

From (\ref{Effpolygon3}) and~(\ref{Effarea})
we know that
\begin{equation}
  \FFP(\q_\parallel,\Gp)=\FFP(0,\Gp)+\mathcal{O}(\q_\parallel).
\end{equation}
Combining this with~(\ref{EF2F2}),
\begin{equation}
  \FFP(\q,\Gamma)=\FFP(0,\Gamma)+\mathcal{O}(\q).
\end{equation}
Therefore, in lowest non-vanishing order,
\begin{equation}
  F(\q\!\to\!0,\Pi)=\frac{\q^*\mathcal{O}(\q)}{q^2}=\text{const}.
\end{equation}
From the definition~(\ref{Eff3d}),
we know of course that this constant, the limiting value~$F(0)$,
is just the volume of
our shape~$\Pi$.



\index{Shape transform!computation|)}
\index{Form factor!computation|)}
