%%%%%%%%%%%%%%%%%%%%%%%%%%%%%%%%%%%%%%%%%%%%%%%%%%%%%%%%%%%%%%%%%%%%%%%%%%%%%%%%
%%
%%   BornAgain User Manual
%%
%%   homepage:   http://www.bornagainproject.org
%%
%%   copyright:  Forschungszentrum Jülich GmbH 2015
%%
%%   license:    Creative Commons CC-BY-SA
%%   
%%   authors:    Scientific Computing Group at MLZ Garching
%%               C. Durniak, M. Ganeva, G. Pospelov, W. Van Herck, J. Wuttke
%%
%%%%%%%%%%%%%%%%%%%%%%%%%%%%%%%%%%%%%%%%%%%%%%%%%%%%%%%%%%%%%%%%%%%%%%%%%%%%%%%%

%-------------------------------------------------------------------------------
%  Page layout
%-------------------------------------------------------------------------------

\def\myparindent{5ex}
\setlength{\parindent}{\myparindent} % workaround, for colorboxes

\textwidth=410pt
\hoffset=210mm % width of A4
\advance\hoffset by -1\textwidth
\hoffset=0.5\hoffset
\advance\hoffset by -1in
% Now a slight assymmetry to leave more blank on the side of the fold
\evensidemargin=0pt
\oddsidemargin=5pt
\advance\evensidemargin by -1\oddsidemargin

\setlength{\headheight}{15pt}
\setlength{\textheight}{592pt} % default=592pt
\setlength{\marginparwidth}{7em}

\renewcommand{\arraystretch}{1.3}

%-------------------------------------------------------------------------------
%  Fancy page header
%-------------------------------------------------------------------------------

\usepackage{fancyhdr}
\pagestyle{fancy}
\fancyhf{}

\renewcommand{\sectionmark}[1]{\markright{#1}}

\fancyhead[LO,RE]{Page \thepage}
\fancyhead[LE]{\nouppercase{\leftmark}}
\fancyhead[RO]{\nouppercase{\rightmark}}
\fancypagestyle{plain}{%
  \fancyhf{} % clear all header and footer fields
  \headwidth=1.1\textwidth
  \renewcommand{\headrulewidth}{1pt}%{1.5pt}
  \renewcommand{\footrulewidth}{0pt}
  \fancyhead[LO,RE]{Page \thepage}
  \fancyhead[LE]{\nouppercase{\leftmark}}
  \fancyhead[RO]{\nouppercase{\rightmark}}
  \fancyhfoffset[L]{24pt}
  \fancyhfoffset[R]{24pt}
}

%-------------------------------------------------------------------------------
%  Symbols, fonts
%-------------------------------------------------------------------------------

\usepackage{amsmath}
\usepackage{mathtools} % has \coloneqq for :=
% \usepackage{manfnt} % for \dbend
\usepackage{dingbat}

\usepackage[bold-style=ISO]{unicode-math} % must come after ams and symbols

%-------------------------------------------------------------------------------
%  Sectioning
%-------------------------------------------------------------------------------

% Add rubber to white space around chapter header
\makeatletter
\def\@makechapterhead#1{%
  \vspace*{50\p@ plus 5\p@ minus 5\p@}%
  {\parindent \z@ \raggedright \normalfont
    \ifnum \c@secnumdepth >\m@ne
        \huge\bfseries \@chapapp\space \thechapter
        \par\nobreak
        \vskip 20\p@ plus 2\p@ minus 2\p@
    \fi
    \interlinepenalty\@M
    \Huge \bfseries #1\par\nobreak
    \vskip 40\p@ plus 5\p@ minus 5\p@
  }}
\def\@makeschapterhead#1{%
  \vspace*{50\p@ plus 5\p@ minus 5\p@}%
  {\parindent \z@ \raggedright
    \normalfont
    \interlinepenalty\@M
    \Huge \bfseries  #1\par\nobreak
    \vskip 40\p@ plus 5\p@ minus 5\p@
  }}
\makeatother

\setcounter{secnumdepth}{3}
\setcounter{tocdepth}{3}
%\usepackage[toc,page]{appendix}
\usepackage{titlesec}

\newcommand{\mysection}[2]{%
  \sectionmark{#1}%
  \section{#2}%
  \sectionmark{#1}%
}

\newcommand{\mychapter}[2]{
  \setcounter{chapter}{#1}
  \setcounter{section}{0}
  \chapter*{#2}
  \addcontentsline{toc}{chapter}{#2}
}

%-------------------------------------------------------------------------------
%  Index, List of Symbols
%-------------------------------------------------------------------------------

\usepackage{imakeidx}
\makeindex

\usepackage[refpage]{nomencl}
\makenomenclature
\renewcommand{\nomname}{List of Symbols}
  % see nomencl.txt for how to force the ordering of symbols
\def\pagedeclaration#1{, \hyperpage{#1}}%
  
\def\otherchapter#1{
  \clearpage
  \phantomsection
  \addcontentsline{toc}{chapter}{#1}
  \markboth{#1}{#1}}

%-------------------------------------------------------------------------------
%  Floats
%-------------------------------------------------------------------------------

\usepackage{subfigure}

\usepackage{placeins} % defines \FloatBarrier
\usepackage{float}
\usepackage[font={small}]{caption}

%-------------------------------------------------------------------------------
%  Tables, code listings, ...
%-------------------------------------------------------------------------------

%\usepackage{longtable}
%\usepackage{booktabs} % defines \toprule &c for use in tabular
% see http://tex.stackexchange.com/questions/78075/multi-page-with-tabulary
\usepackage{tabulary}

\usepackage[final]{listings}
\usepackage{lstcustom}
\renewcommand{\lstfontfamily}{\ttfamily}

%-------------------------------------------------------------------------------
%  Tikz pictures
%-------------------------------------------------------------------------------

\usepackage{tikz}
%\usepackage{tikz-uml} 
\usetikzlibrary{trees,matrix,positioning,decorations.pathreplacing,calc}

\newcommand{\ntikzmark}[2]
           {#2\thinspace\tikz[overlay,remember picture,baseline=(#1.base)]
             {\node[inner sep=0pt] (#1) {};}}

\newcommand{\makebrace}[3]{%
    \begin{tikzpicture}[overlay, remember picture]
        \draw [decoration={brace,amplitude=0.6em},decorate]
        let \p1=(#1), \p2=(#2) in
        ({max(\x1,\x2)}, {\y1+1.5em}) -- node[right=0.6em] {#3} ({max(\x1,\x2)}, {\y2});
    \end{tikzpicture}
}

%-------------------------------------------------------------------------------
%  Highlighting
%-------------------------------------------------------------------------------

\usepackage{ifdraft}
\usepackage{mdframed}

\def\defineBox#1#2#3#4#5{
  \newmdenv[
    usetwoside=false,
    skipabove=3pt minus 1pt plus 3pt,
    skipbelow=3pt minus 1pt plus 3pt,
    leftmargin=-4pt,
    rightmargin=-4pt,
    innerleftmargin=2pt,
    innerrightmargin=2pt,
    innertopmargin=4pt,
    innerbottommargin=4pt,
    backgroundcolor=#3,
    topline=false,
    bottomline=false,
    linecolor=#4,
    linewidth=2pt,
%    nobreak=true,
    ]{#2*}
  \newenvironment{#1}
    {\begin{#2*}\makebox[0pt][r]{\smash{#5}}\ignorespaces}
    {\end{#2*}}
}

\def\marginSymbolLarge#1#2{\ifdraft{\textbf{#2~~~~}}{\raisebox{-3ex}%
{\includegraphics[width=3em]{#1}\hspace{10pt}}}}

\defineBox{boxWork}{boxxWork}{magenta!40}{magenta}
  {\marginSymbolLarge{fig/fancymanual/Arbeiten.png}{TODO}}
\defineBox{boxWarn}{boxxWarn}{magenta!40}{magenta}
  {\marginSymbolLarge{fig/fancymanual/Achtung.png}{WARN}}
\defineBox{boxNote}{boxxNote}{yellow!33}{yellow}{{}}
\defineBox{boxEmph}{boxxEmph}{green!20}{green}{{}}

\def\Warn#1{\begin{boxWarn}#1\end{boxWarn}}
\def\Work#1{\begin{boxWork}#1\end{boxWork}}
\def\Note#1{\begin{boxNote}#1\end{boxNote}}
\def\Emph#1{\begin{boxEmph}#1\end{boxEmph}}

\def\MissingSection{\begin{boxWork}\ldots\ to be written \ldots\end{boxWork}}

% OLD STYLE:
  
\newcommand{\BareRemark}[1]%
{\noindent\smallpencil\colorbox{blue!10}%
{\parbox{\dimexpr\linewidth-8\fboxsep}{#1}}}

\newcommand{\MakeRemark}[2]{\BareRemark{\underline{#1} #2 }}

\newcommand{\ImportantPoint}[2]
{\noindent
  {\huge\danger}\colorbox{magenta!40}{\parbox{\dimexpr\linewidth-8\fboxsep}
 {\underline{#1} #2}}}


%-------------------------------------------------------------------------------
%  Hyperref
%-------------------------------------------------------------------------------

\usepackage[final]{hyperref} % wants to be included last
\hypersetup{
    colorlinks,
    linkcolor={red!50!black},
    citecolor={blue!50!black},
    urlcolor={blue!80!black},
    pdftitle={BornAgain User Manual} % seems to be ignored
}
\ifdraft{\usepackage[right]{showlabels}}{}
