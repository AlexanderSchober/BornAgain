%%%%%%%%%%%%%%%%%%%%%%%%%%%%%%%%%%%%%%%%%%%%%%%%%%%%%%%%%%%%%%%%%%%%%%%%%%%%%%%%
%%
%%   BornAgain User Manual
%%
%%   homepage:   http://www.bornagainproject.org
%%
%%   copyright:  Forschungszentrum Jülich GmbH 2015
%%
%%   license:    Creative Commons CC-BY-SA
%%   
%%   authors:    Scientific Computing Group at MLZ Garching
%%               C. Durniak, M. Ganeva, G. Pospelov, W. Van Herck, J. Wuttke
%%
%%%%%%%%%%%%%%%%%%%%%%%%%%%%%%%%%%%%%%%%%%%%%%%%%%%%%%%%%%%%%%%%%%%%%%%%%%%%%%%%

\let\angstrom=\AA

%-------------------------------------------------------------------------------
%  Page layout
%-------------------------------------------------------------------------------

% Horizontal setup
\textwidth=410pt
\hoffset=210mm % width of A4
\advance\hoffset by -1\textwidth
\hoffset=0.5\hoffset
\advance\hoffset by -1in
% Now a slight assymmetry to leave more blank on the side of the fold
\evensidemargin=0pt
\oddsidemargin=5pt
\advance\evensidemargin by -1\oddsidemargin

\def\myparindent{5ex}
\setlength{\parindent}{\myparindent} % workaround, for colorboxes

% Vertical setup
\setlength{\headheight}{0pt}
\setlength{\headsep}{10pt}
\setlength{\textheight}{630pt} % default=592pt
\setlength{\footskip}{45pt}
\setlength{\marginparwidth}{7em}
\renewcommand{\baselinestretch}{1.02}

\renewcommand{\arraystretch}{1.3}

%-------------------------------------------------------------------------------
%  Symbols, fonts
%-------------------------------------------------------------------------------

\usepackage{amsmath}
\usepackage{mathtools} % has \coloneqq for :=
% \usepackage{manfnt} % for \dbend
\usepackage{dingbat}
\usepackage{amssymb}

\usepackage[bold-style=ISO]{unicode-math} % must come after ams and symbols
% prevent unicode-math from overwriting
\AtBeginDocument{\renewcommand{\Re}{\operatorname{Re}}}
\AtBeginDocument{\renewcommand{\Im}{\operatorname{Im}}}

% Math operators 
\DeclareMathOperator{\sinc}{sinc}

%-------------------------------------------------------------------------------
%  Sectioning
%-------------------------------------------------------------------------------

% Add rubber to white space around chapter header
\makeatletter
\newif\ifnumberedchapter
\def\@makechapterhead#1{\numberedchaptertrue\mychapterhead{#1}}
\def\@makeschapterhead#1{\numberedchapterfalse\mychapterhead{#1}}

\def\mychapterhead#1{%
  \vspace*{50\p@ plus 10\p@ minus 10\p@}%
  {\parindent \z@ \normalfont
    \ifnumberedchapter
    \raggedright
    \Large\bfseries \@chapapp\space \thechapter
    \par\nobreak
    \vskip 20\p@ plus 4\p@ minus 4\p@
    \interlinepenalty\@M
    \fi
%    \hrule
    \vskip 10\p@ plus 2\p@ minus 2\p@
    \interlinepenalty\@M
    \raggedright
  \huge \bfseries #1\par\nobreak
    \interlinepenalty\@M
    \vskip 10\p@ plus 2\p@ minus 2\p@
%    \hrule
    \interlinepenalty\@M
    \vskip 40\p@ plus 8\p@ minus 8\p@
  }}

\def\otherchapter#1{
  \clearpage
  \phantomsection
  \addcontentsline{toc}{chapter}{#1}
  \markboth{#1}{#1}}

\def\ichapter#1{\chapter*{#1}\addcontentsline{toc}{chapter}{#1}}
\def\isection#1{\section*{#1}\addcontentsline{toc}{section}{#1}}

\renewcommand\section{\@startsection {section}{1}{\z@}%
                                   {-3.5ex \@plus -1.5ex \@minus -.5ex}%
                                   {2.3ex \@plus .8ex \@minus .5ex}%
                                   {\normalfont\Large\bfseries}}
\renewcommand\subsection{\@startsection{subsection}{2}{\z@}%
                                     {-3.25ex\@plus -1.3ex \@minus -.4ex}%
                                     {1.5ex \@plus .5ex \@minus .3ex}%
                                     {\normalfont\large\bfseries}}
% from size11.clo
\renewcommand\normalsize{
   \@setfontsize\normalsize\@xipt{13.6}%
   \abovedisplayskip 11\p@ \@plus7\p@ \@minus6\p@
   \abovedisplayshortskip \z@ \@plus3\p@
   \belowdisplayshortskip 6.5\p@ \@plus3.5\p@ \@minus3\p@
   \belowdisplayskip \abovedisplayskip
   \let\@listi\@listI}
\makeatother

\setcounter{secnumdepth}{3}
\setcounter{tocdepth}{3}
%\usepackage[toc,page]{appendix}
\usepackage{titlesec}

%-------------------------------------------------------------------------------
%  Index, List of Symbols
%-------------------------------------------------------------------------------

\usepackage{imakeidx}
\makeindex

\usepackage[refpage]{nomencl}
\makenomenclature
\renewcommand{\nomname}{List of Symbols}
  % see nomencl.txt for how to force the ordering of symbols
  
%-------------------------------------------------------------------------------
%  Improve LaTeX basics
%-------------------------------------------------------------------------------

\usepackage{enumitem}
\usepackage{subfigure}

\usepackage{placeins} % defines \FloatBarrier
\usepackage{float}
\usepackage[font={small}]{caption}

%-------------------------------------------------------------------------------
%  Tables, code listings, ...
%-------------------------------------------------------------------------------

\usepackage{longtable}
%\usepackage{booktabs} % defines \toprule &c for use in tabular
% see http://tex.stackexchange.com/questions/78075/multi-page-with-tabulary
\usepackage{tabulary}

\usepackage[final]{listings}
\usepackage{lstcustom}
\renewcommand{\lstfontfamily}{\ttfamily}
\lstset{language=python,style=eclipseboxed,numbers=none,nolol}

%-------------------------------------------------------------------------------
%  Tikz pictures
%-------------------------------------------------------------------------------

\usepackage{tikz}
%\usepackage{tikz-uml} 
\usetikzlibrary{trees,matrix,positioning,decorations.pathreplacing,calc}

\newcommand{\ntikzmark}[2]
           {#2\thinspace\tikz[overlay,remember picture,baseline=(#1.base)]
             {\node[inner sep=0pt] (#1) {};}}

\newcommand{\makebrace}[3]{%
    \begin{tikzpicture}[overlay, remember picture]
        \draw [decoration={brace,amplitude=0.6em},decorate]
        let \p1=(#1), \p2=(#2) in
        ({max(\x1,\x2)}, {\y1+1.5em}) -- node[right=0.6em] {#3} ({max(\x1,\x2)}, {\y2});
    \end{tikzpicture}
}

%-------------------------------------------------------------------------------
%  Highlighting
%-------------------------------------------------------------------------------

\usepackage{ifdraft}
\usepackage{mdframed}
\input FixMdframed % bug fix to prevent erroneous page breaks
% doesnt work:
%\newcommand\widow{%
%  \widowpenalty=10000
%
%  \widowpenalty=150}

\def\defineBox#1#2#3#4#5{
  \newmdenv[
    usetwoside=false,
    skipabove=3pt minus 1pt plus 3pt,
    skipbelow=3pt minus 1pt plus 3pt,
    leftmargin=-4pt,
    rightmargin=-4pt,
    innerleftmargin=2pt,
    innerrightmargin=2pt,
    innertopmargin=4pt,
    innerbottommargin=4pt,
    backgroundcolor=#3,
    topline=false,
    bottomline=false,
    linecolor=#4,
    linewidth=2pt,
    ]{#2*}
  \newenvironment{#1}
    {\begin{#2*}\makebox[0pt][r]{\smash{#5}}\ignorespaces}
    {\end{#2*}\mdbreakon}
}

\def\mdbreakoff{\makeatletter\booltrue{mdf@nobreak}\makeatother}
\def\mdbreakon{\makeatletter\boolfalse{mdf@nobreak}\makeatother}

\def\marginSymbolLarge#1#2{\ifdraft{\textbf{#2~~~~}}{\raisebox{-3ex}%
{\includegraphics[width=3em]{#1}\hspace{10pt}}}}

\defineBox{boxWork}{boxxWork}{magenta!40}{magenta}
  {\marginSymbolLarge{fig/fancymanual/Arbeiten.png}{TODO}}
\defineBox{boxWarn}{boxxWarn}{magenta!40}{magenta}
  {\marginSymbolLarge{fig/fancymanual/Achtung.png}{WARN}}
\defineBox{boxNote}{boxxNote}{yellow!33}{yellow}{{}}
\defineBox{boxTuto}{boxxTuto}{blue!20}{blue}{{}}
\defineBox{boxEmph}{boxxEmph}{green!20}{green}{{}}

\def\Warn#1{\begin{boxWarn}#1\end{boxWarn}}
\def\Work#1{\begin{boxWork}#1\end{boxWork}}
\def\Note#1{\begin{boxNote}#1\end{boxNote}}
\def\Tuto#1{\begin{boxTuto}#1\end{boxTuto}}
\def\Emph#1{\begin{boxEmph}#1\end{boxEmph}}

\def\MissingSection{\begin{boxWork}\ldots\ to be written \ldots\end{boxWork}}

% % OLD STYLE:
%   
% \newcommand{\BareRemark}[1]%
% {\noindent\smallpencil\colorbox{blue!10}%
% {\parbox{\dimexpr\linewidth-8\fboxsep}{#1}}}
% 
% \newcommand{\MakeRemark}[2]{\BareRemark{\underline{#1} #2 }}
% 
% \newcommand{\ImportantPoint}[2]
% {\noindent
%   {\huge\danger}\colorbox{magenta!40}{\parbox{\dimexpr\linewidth-8\fboxsep}
%  {\underline{#1} #2}}}


%-------------------------------------------------------------------------------
%  Hyperref
%-------------------------------------------------------------------------------

\usepackage[final]{hyperref} % wants to be included last
\hypersetup{
    colorlinks,
    linkcolor={red!50!black},
    citecolor={blue!50!black},
    urlcolor={blue!80!black},
    pdftitle={BornAgain User Manual} % seems to be ignored
}
\def\tuto#1#2{\href{http://bornagainproject.org/node/#1}{#2}}
\ifdraft{\usepackage[right]{showlabels}}{}
