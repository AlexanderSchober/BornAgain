%%%%%%%%%%%%%%%%%%%%%%%%%%%%%%%%%%%%%%%%%%%%%%%%%%%%%%%%%%%%%%%%%%%%%%%%%%%%%%%%
%%
%%   BornAgain:  simulate and fit scattering at grazing incidence
%%
%%   homepage:   http://www.bornagainproject.org
%%
%%   copyright:  Forschungszentrum Jülich GmbH 2015
%%               Software licence does not cover this documentation
%%               For documentation licence, contact the authors
%%   
%%   authors:    Scientific Computing Group at MLZ Garching
%%               C. Durniak, M. Ganeva, G. Pospelov, W. Van Herck, J. Wuttke
%%
%%%%%%%%%%%%%%%%%%%%%%%%%%%%%%%%%%%%%%%%%%%%%%%%%%%%%%%%%%%%%%%%%%%%%%%%%%%%%%%%


\chapter{Theory} \label{appendixtheory}

\section{Scattering on nanoparticles - Formal treatment} \label{sec:formal}

\subsection{Green operators and the $T$--matrix}

For a particle, governed by the Schr\"odinger equation with Hamiltonian $H = H_0 + V$, the time--independent scattering theory formally consists of solving the eigenvalue equations:

\begin{equation*}
  H\Psi_\alpha = E_\alpha\Psi_\alpha \; ,
\end{equation*}
with $E$ the scalar energy eigenvalue of the eigenstate $\Psi(E)$.\\
If the solutions of the free (or unperturbed) Hamiltonian $H_0$ are known:
\begin{equation*}
  H_0\Psi_{0\alpha} = E_\alpha\Psi_{0\alpha} \; ,
\end{equation*}
one can write the solutions of the full Hamiltonian in terms of these asymptotic states and Green operators:

\begin{align*}
  \Psi^\pm_\alpha &= \Psi_{0\alpha} + G^\pm_0 V \Psi^\pm_\alpha  \\
  & = \Psi_{0\alpha} + G^\pm V  \Psi_{0\alpha} \; ,
\end{align*}

where the Green operators are defined as:
\begin{align*}
  G^\pm_0(E) &= (E-H_0\pm i\epsilon )^{-1}  \\
  G^\pm (E) &= (E-H\pm i\epsilon )^{-1} \; .
\end{align*}
In these equations, the upper index or sign refers to the state corresponding with the free state $\Psi_{0\alpha}$ at time $t\rightarrow - \infty$ (and vice--versa for the lower sign). Since the solutions of the eigenvalue equations, both for the unperturbed as for the full Hamiltonian, are dependent on the energy eigenvalue $E$, the index $\alpha$ is assumed to include this value (and possibly other quantum numbers).

The transition amplitude between two asymptotic states is given by the $S$--matrix elements, defined as:
\begin{align}
  S_{\alpha\beta} &\equiv \braket{\Psi_{0\beta}}{S}{\Psi_{0\alpha}} \nonumber\\
  & \equiv \braketnoop{\Psi_\beta^-}{\Psi_\alpha^+} \; .
  \label{<++>}
\end{align}

The $S$--matrix can be decomposed into a delta function, representing the absence of scattering, and a $T$--matrix that encodes the scattering part, caused by the potential $V$:
\begin{equation*}
  S_{\alpha\beta} = \delta (E_\alpha - E_\beta)\delta_{\alpha\beta} - 2\pi i \delta (E_\alpha - E_\beta) T^\pm_{\alpha\beta} \; ,
\end{equation*}
with
\begin{align*}
  T^+_{\alpha\beta} & = \braket{\Psi_{0\beta}}{V}{\Psi^+_\alpha} \\
  T^-_{\alpha\beta} & = \braket{\Psi^-_\beta}{V}{\Psi_{0\alpha}} \; .
\end{align*}
On the energy shell $E_\alpha = E_\beta$, one has $T^+_{\alpha\beta}= T^-_{\alpha\beta}$, so that both formulations are equivalent.

By expanding the eigenstates $\Psi^\pm_\alpha$ in these equations, the $T$--matrix elements (on--shell) can be expressed as:
\begin{equation*}
  T^\pm_{\alpha\beta} = V + VG^+ V \; .
\end{equation*}


\subsection{Momentum representation and the scattering cross--section}
The previous general formulas can also be presented in a momentum (and positon) eigenbasis, defined by:
\begin{align*}
  \oper{P}\ket{\vect{k}} & = \hbar \vect{k} \ket{\vect{k}} \\
  \braketnoop{\vect{k'}}{\vect{k}} & = \delta(\vect{k'}-\vect{k}) \\
  1 &= \int d^3\vect{k} \ket{\vect{k}} \bra{\vect{k}} \\
  1 &= \int d^3\vect{r} \ket{\vect{r}} \bra{\vect{r}} \\
  \braketnoop{\vect{r}}{\vect{k}} &= (2\pi)^{-3/2} \exp (i\vect{k}\cdot\vect{r}) \; ,
\end{align*}
where the normalization in the last equation follows from the other definitions.

The wavefunction that evolves from a momentum eigenstate $\ket{\vect{k}_i}$ can then be written as:
\begin{equation*}
  \braketnoop{\vect{r}}{\Psi^+} = \braketnoop{\vect{r}}{\vect{k}_i} + \braket{\vect{r}}{G_0^+T}{\vect{k}_i} \; ,
\end{equation*}
which in the far--field limit becomes:
\begin{align*}
  \braketnoop{\vect{r}}{\Psi^+} &= (2\pi)^{-3/2}\left[ e^{i\vect{k}_i\cdot\vect{r}} - \frac{4\pi^2m}{\hbar^2}\cdot\frac{e^{ik_f r}}{r} \braket{\vect{k}_f}{T}{\vect{k}_i} \right]  \\
  &\equiv (2\pi)^{-3/2}\left[ e^{i\vect{k}_i\cdot\vect{r}} + f(\theta,\phi)\frac{e^{ik_f r}}{r} \right] \; ,
\end{align*}
where the scattering amplitude was defined as
\begin{equation*}
  f(\theta, \phi) = -\frac{4\pi^2m}{\hbar^2}\braket{\vect{k}_f}{T}{\vect{k}_i} \; .
\end{equation*}

The amount of particles per unit time that are scattered in a small solid angle $d\Omega$ in direction $\vect{k}_f$ will then be (still in the far--field limit):
\begin{equation*}
  dI_{scat} = J_0\lvert f(\theta,\phi)\rvert^2d\Omega \; ,
\end{equation*}
where $J_0$ denotes the incident flux density. The scattering cross--section is defined as:
\begin{equation*}
  \frac{d\sigma}{d\Omega}\equiv \frac{dI_{scat}}{J_0 d\Omega} = \lvert f(\theta,\phi) \rvert^2 \; .
\end{equation*}
%%%%%%%%%%%%%%%%%%%%%%%%%%%%%%%%%%%%%%%%%%%
\section{Small angle approximation}
In the case of pure nuclear scattering, the Hamiltonian describing a neutron in a scattering experiment, is given by $H = -\dfrac{\hbar^2}{2m}\Delta + V$, where

\begin{equation*}
  V = \frac{2\pi \hbar^2}{m}\rho_s(\vect{r}) \; ,
\end{equation*}
with $\rho_s(\vect{r})$ the scattering length density of the sample. This scattering length density typically consists of a sum of weighted delta--functions, peaked at the atomic postions of the sample. For small scattering angles, the Bragg condition will not be fulfilled and the scattering length density may be replaced by a continuous function, representing the average scattering length density. In this case, one can define a refractive index, which in general will also be a continuous function of the position in the sample:
\begin{equation*}
  n^2(\vect{r}) \equiv 1 - \frac{4\pi}{k_0^2}\rho_s(\vect{r}) \; ,
\end{equation*}
with $k_0$ the wavevector in vacuum, or alternatively $k_0 = 2\pi /\lambda$, with $\lambda$ the de Broglie wavelength of the neutron.


Substituting this refractive index in the potential then gives:
\begin{equation*}
  V(\vect{r}) = \frac{\hbar^2}{2m}k_0^2(1-n^2(\vect{r})) \; .
\end{equation*}

Using these definitions, one can rescale the Hamiltonian with a factor $2m/\hbar^2$, such that
\begin{align*}
  \widetilde{H} &\equiv -\Delta + \widetilde{V} \nonumber \\
  \widetilde{V}(\vect{r}) &\equiv 4\pi\rho_s(\vect{r}) = k_0^2(1-n^2(\vect{r})) \; .
\end{align*}
It should be noted that this Hamiltonian implicitly contains the energy eigenvalue ($E_{k_0}=(\hbar k_0)^2/2m$), so that it can only be used in the time--independent Schr\"odinger equation $H\Psi_\alpha = E_\alpha \Psi_\alpha$.

The $T$--matrix then also becomes rescaled and the scattering amplitude becomes:
\begin{equation*}
  f(\theta, \phi) = -2\pi^2 \braket{\vect{k}_f}{\widetilde{T}}{\vect{k}_i} \; .
\end{equation*}

%%%%%%%%%%%%%%%%%%%%%%%%%%%%%%%%%%
\section{Born approximation} \label{sec:ba}
Consider a scattering volume $V$, containing $N$ scattering centers with shape functions $S^i(\vect{r})$, positions $\vect{R}^i$ and scattering length density $\rho_s$ (relative to the ambient material).

In the Born approximation ($\widetilde{T}\simeq\widetilde{V}$), the scattering amplitude is
\begin{align*}
  f(\theta, \phi) & = -8\pi^3 \braket{\vect{k_f}}{\rho_s(\vect{r})}{\vect{k_i}} \nonumber \\
  & = -\int d^3\vect{r} e^{i\vect{q}\cdot\vect{r}} \rho_s(\vect{r}) \; ,
\end{align*}
where $\vect{q}\equiv \vect{k}_i - \vect{k}_f$ denotes the wavevector transfer and
\begin{equation*}
  \rho_s(\vect{r}) = \frac{k_0^2}{4\pi}(1-n^2(\vect{r})) \; .
\end{equation*}

The differential cross--section (per scattering center) is then given by:
\begin{equation*}
  \frac{d\sigma}{d\Omega}(\vect{q}) = \frac{1}{N}\left\lvert \int_V \rho_s(\vect{r}) e^{i\vect{q}\cdot\vect{r}} d^3\vect{r} \right\rvert ^2 \; .
\end{equation*}

Following the initial assumptions. the scattering length density can be written as:
\begin{equation*}
\rho_s(\vect{r}) = \sum_i \rho_{s,i} S^i(\vect{r}) \otimes \delta (\vect{r}-\vect{R}^i) \; ,
\end{equation*}
with $\rho_{s,i}$ the scattering length density of particle $i$. The cross--section then becomes:
\begin{align*}
  N\frac{d\sigma}{d\Omega}(\vect{q}) & = \left\lvert \sum_i F^i(\vect{q}) \exp (i\vect{q}\cdot\vect{R}^i) \right\rvert ^2  \\
  & = \left\lbrace \sum_i \left\lvert F^i(\vect{q}) \right\rvert ^2 + \sum_{i\neq j} F^i(\vect{q}) F^{j*}(\vect{q}) \exp \left[i\vect{q}\cdot (\vect{R}^i-\vect{R}^j)\right] \right\rbrace \; .
\end{align*}
In the last expression, the formfactors $F^i(\vect{q})$ are the Fourier transforms of the shape functions, including their scattering length densities:
\begin{equation*}
  F^i(\vect{q}) \equiv \int_V d^3\vect{r} \rho_{s,i} S^i(\vect{r}) \exp (i\vect{q}\cdot\vect{r} ) \; .
\end{equation*}


Since in most real conditions only the statistical properties of the particles are known, one can consider the expectation value of this cross--section. Assuming that the particles' shapes are determined by their class $\alpha$, with abundance ratio $p_\alpha \equiv N_\alpha / N$, and defining the particle density $\rho_V \equiv N/V$, the expectation value becomes:
\begin{align*}
  \left\langle \frac{d\sigma}{d\Omega}(\vect{q}) \right\rangle  & = \sum_\alpha p_\alpha \left\lvert F_\alpha(\vect{q})\right\rvert ^2 + \frac{\rho_V}{V}\sum_{\alpha,\beta} p_\alpha p_\beta F_\alpha (\vect{q})F_\beta^*(\vect{q})  \\
  & \times \iint_V d^3\vect{R}_\alpha d^3\vect{R}_\beta \ppcf{\alpha}{\beta}{R} \exp \left[ i\vect{q}\cdot (\vect{R}_\alpha - \vect{R}_\beta ) \right] \; .
\end{align*}

In this equation, the factor $\ppcf{\alpha}{\beta}{R}$ is called the \emph{partial pair correlation function} and it represents a normalized probability of finding particles of type $\alpha$ and $\beta$ in positions $\vect{R}_\alpha$ and $\vect{R}_\beta$ respectively. More precisely, the probability density for finding a particle $\alpha$ at position $\vect{R}_\alpha$ and another one of type $\beta$ at $\vect{R}_\beta$ is given by:
\begin{equation*}
  \mathcal{P}(\alpha, \vect{R}_\alpha ; \beta , \vect{R}_\beta ) \equiv \rho_V^2 p_\alpha p_\beta \ppcf{\alpha}{\beta}{R} \; .
\end{equation*}

\subsection{General formulas}
Even in the most general case, the partial pair correlation function will only depend on the difference $\vect{R}_{\alpha\beta}\equiv(\vect{R}_\alpha - \vect{R}_\beta )$ of the particles' positions. One of the volume integrals can then be dropped, together with the volume factor, giving:
\begin{align*}
  \left\langle \frac{d\sigma}{d\Omega}(\vect{q}) \right\rangle  & = \sum_\alpha p_\alpha \left\lvert F_\alpha(\vect{q})\right\rvert ^2 + \rho_V\sum_{\alpha,\beta} p_\alpha p_\beta F_\alpha (\vect{q})F_\beta^*(\vect{q}) \\
  & \times \int_V d^3\vect{R}_{\alpha\beta} \ppcfb{\alpha}{\beta}{R} \exp \left[ i\vect{q}\cdot \vect{R}_{\alpha\beta} \right] \; .
\end{align*}

This expression can be split into a diffuse part, which by definition should be zero for the case of only one particle type, and a coherent part, resulting from the coherent superposition of scattering amplitudes for particles at different positions:
\begin{equation*}
  \left\langle \frac{d\sigma}{d\Omega}(\vect{q}) \right\rangle = I_d(\vect{q}) + \ensavg{\alpha\beta}{F_\alpha (\vect{q} ) S_{\alpha\beta} (\vect{q}) F_\beta^* (\vect{q})} \; ,
\end{equation*}
where
\begin{align*}
  I_d(\vect{q}) &\equiv \ensavg{\alpha}{\left\rvert F_\alpha (\vect{q}) \right\rvert^2} - \left\lvert \ensavg{\alpha}{ F_\alpha (\vect{q})} \right\rvert^2 \; , \\
  S_{\alpha\beta} (\vect{q}) &\equiv 1 + \rho_V \int_V d^3\vect{R}_{\alpha\beta}\ppcfb{\alpha}{\beta}{R} \exp \left[ i\vect{q}\cdot \vect{R}_{\alpha\beta} \right] \; .
\end{align*}
$S_{\alpha\beta} (\vect{q})$ is called the \emph{interference function} and $\langle\dotso\rangle_\alpha$ is the expectation value over the classes $\lbrace \alpha\rbrace$.

\subsection{Decoupling approximation}
When the partial pair correlation function is independent of the particle class $\alpha$ ($ \ppcfb{\alpha}{\beta}{R} \equiv g(\vect{R}_{\alpha\beta})$ ), the scattering cross--section becomes:
\begin{equation*}
\left\langle \frac{d\sigma}{d\Omega}(\vect{q}) \right\rangle  = I_d(\vect{q}) + \left\lvert \left\langle F_\alpha(\vect{q}) \right\rangle_\alpha \right\rvert ^2 \times S(\vect{q}) \; ,
\end{equation*}
where
\begin{equation*} 
  S(\vect{q}) = 1+ \rho_V \int_V d^3\vect{R} \; g(\vect{R}) \exp \left[ i\vect{q}\cdot \vect{R} \right] \; .
\end{equation*}

\subsection{Local Monodisperse Approximation}
By assuming that inside every coherence region of the beam, the particle class (or size/shape) is fixed, the cross--section will consist of an incoherent superposition of these different coherence regions and can be written as:
\begin{equation*}
  \left\langle \frac{d\sigma}{d\Omega}(\vect{q}) \right\rangle \simeq \left\langle \left\lvert F_\alpha(\vect{q})\right\rvert ^2 S_\alpha(\vect{q}) \right\rangle_\alpha \; .
\end{equation*}

Contrary to the Decoupling Approximation, the Local Monodisperse Approximation can account for particle class/size/shape--dependent pair correlation functions by having distinct interference functions $S_\alpha(\vect{q})$.

%%%%%%%%%%%%%%%%%%%%%%%%%%%%%%%%%%%%%%%%%%%
\subsection{Size--Spacing Correlation Approximation}
In the Size--Spacing Correlation Approximation, a correlation is assumed between the shape/size of the particles and their mutual spacing. A classical example would consist of particles whose closest--neighbour spacing depends linearly on the sum of their respective sizes. The following discussion of this type of correlation is inspired by \cite{LaLR07}

The scattered intensity can also be calculated as the Fourier transform of the Patterson function, which is the autocorrelation of the scattering length density:
\begin{equation*}
  \curlp (\vectr ) \equiv \sum_{ij} S_i(-\vectr )\otimes S_j(\vectr )\otimes \delta (\vectr + \vectr_i - \vectr_j ) \; .
\end{equation*}
For a sample where only the statistical properties of particle positions and shape/size are known, the scattered intensity per scattering particle becomes average over an ensemble of the Fourier transform of the Patterson function:
\begin{equation*}
  I(\vectq ) = \frac{1}{N}\ensavg{}{\curlf (\curlp (\vectr ))} \; ,
\end{equation*}
where $\curlf$ denotes the Fourier transform.

The ensemble averaged Patterson function will be denoted as:
\begin{equation*}
  Z(r) \equiv \frac{1}{N}\ensavg{}{\curlp (\vectr )} \; .
\end{equation*}
In the case of systems where the particles are aligned in one dimension, this autocorrelation function can be further split into nearest neighbour probabilities. First, it is split into terms for negative, zero or positive distance:
\begin{equation*}
  Z(\vectr ) \equiv z_0(\vectr ) + z_+(\vectr ) + z_-(\vectr ) \; .
\end{equation*}
Taking $x$ as the coordinate in the direction in which the particles are arranged and $s$ as an orthogonal coordinate ($\vectr \equiv (x,s)$), one obtains:
\begin{align*}
  z_0(\vectr ) &= \sum_{\alpha_0} p(\alpha_0) S_{\alpha_0}(-x,-s) \otimes S_{\alpha_0}(x,s)  \\
  z_+(\vectr ) &= \sum_{\alpha_0\alpha_1} p(\alpha_0,\alpha_1) S_{\alpha_0}(-x,-s) \otimes S_{\alpha_1}(x,s) \otimes P_1(x|\alpha_0\alpha_1)  \\
               &+ \sum_{\alpha_0\alpha_1\alpha_2} p(\alpha_0,\alpha_1,\alpha_2) S_{\alpha_0}(-x,-s) \otimes S_{\alpha_2}(x,s) \otimes P_1(x|\alpha_0\alpha_1) \otimes P_2(x|\alpha_0\alpha_1\alpha_2)  \\
               &+ \dotsb \\
  z_-(\vectr ) &= z_+(-\vectr ) \; ,
\end{align*}
where $p(\alpha_0,\dotsc ,\alpha_n)$ denotes the probability of having a sequence of particles of the indicated sizes/shapes and $P_n(x|\alpha_0\dotsc\alpha_n)$ is the probability density of having a particle of type $\alpha_n$ at a (positive) distance $x$ of its nearest neighbour of type $\alpha_{n-1}$ in a sequence of the given order.



In the Size--Spacing Correlation Approximation, one assumes that the particle sequence probabilities are just a product of their individual fractions:
\begin{equation*}
  p(\alpha_0,\dotsc ,\alpha_n) = \prod_i p(\alpha_i) \; ,
\end{equation*}
and the nearest neighbour distance distribution is dependent only on the two particles involved:
\begin{equation*}
  P_n(x|\alpha_0\dotsc\alpha_n) = P_1(x|\alpha_{n-1}\alpha_n) \; .
\end{equation*}
Furthermore, the distance distribution $P_1(x|\alpha_0\alpha_1)$ depends on the particle sizes/shapes only through its mean value $D$:
\begin{equation*}
  P_1(x|\alpha_0\alpha_1) = P_0(x - D(\alpha_0,\alpha_1) ) \; ,
\end{equation*}
where $D(\alpha_0,\alpha_1) = D_0 + \kappa \left[ \Delta R(\alpha_0) + \Delta R(\alpha_1) \right]$, with $\Delta R(\alpha_i)$ the deviation of a size parameter of particle $i$ with respect to the mean over all particles sizes/shapes and $\kappa$ the coupling parameter.

In momentum space, the sum of convolutions can be written as a geometric series, which can be exactly calculated to be:
\begin{equation}
\label{eq:sscainf}
  I(\vectq ) = \ensavg{\alpha}{\left| F_\alpha(\vectq ) \right| ^2} + 2 \Re \left\lbrace \widetilde{\curlf_\kappa}(\vectq )\widetilde{\curlf_\kappa^*}(\vectq ) \cdot \frac{\Omega_\kappa(\vectq )}{\tilde{p}_{2\kappa}(\vectq )\left[ 1 - \Omega_\kappa(\vectq )\right] } \right\rbrace \; ,
\end{equation}
with
\begin{align*}
  \tilde{p}_\kappa(\vectq ) &= \int d\alpha\; p(\alpha) e^{i\kappa q_x \Delta R(\alpha)}  \\
  \Omega_\kappa(\vectq ) &= \tilde{p}_{2\kappa}(\vectq ) \phi(\vectq) e^{i q_x D_0}  \\
  \widetilde{\curlf_\kappa}(\vectq ) &= \int d\alpha\; p(\alpha)F_\alpha (\vectq ) e^{i\kappa q_x \Delta R(\alpha)} \; ,
\end{align*}
and the Fourier transform of $P_1(x|\alpha_0\alpha_1)$ is
\begin{equation*}
  \curlp (\vectq ) = \phi (\vectq )e^{i q_x D_0} e^{i \kappa q_x \left[ \Delta R(\alpha_0) + \Delta R(\alpha_1) \right] } \; .
\end{equation*}

Using the result from the one--dimensional analysis, one can apply this formula ad hoc for distributions of particles in a plane, where the coordinate $x$ will now be replaced with $(x,y)$, while the $s$ coordinate encodes a position in the remaining orthogonal direction. One must be aware however that this constitutes a further approximation, since this type of correlation does not have a general solution in more than one dimension.

The intensity in \refeq{sscainf} will contain a Dirac delta function contribution, caused by taking an infinite sum of terms that are perfectly correlated at $\vectq = 0$. One can leverage this behaviour by multiplying the nearest neighbour distribution by a constant factor $e^{-D/\Lambda}$, which removes the division by zero in \refeq{sscainf}.
Another way of dealing with this infinity at $\vectq =0$ consists of taking only a finite number of terms, in which case the geometric series still has an analytical solution, but becomes a bit more cumbersome:
\begin{equation*}
\begin{split}
  I(\vectq ) &= \ensavg{\alpha}{\left| F_\alpha(\vectq ) \right| ^2} + 2 \Re \Biggl\lbrace \frac{1}{\tilde{p}_{2\kappa}(\vectq )}\widetilde{\curlf_\kappa}(\vectq )\widetilde{\curlf_\kappa^*}(\vectq ) \\
  & \times \left[ \left( 1 - \frac{1}{N}\right) \frac{\Omega_\kappa(\vectq )}{1 - \Omega_\kappa(\vectq ) } - \frac{1}{N}\frac{\Omega_\kappa^2(\vectq )\left( 1- \Omega_\kappa^{N-1}(\vectq )\right) }{\left( 1 - \Omega_\kappa(\vectq ) \right) ^2 } \right] \Biggr\rbrace \; .
\end{split}
\end{equation*}
This expression has a well--defined limit for $\Omega_\kappa(\vectq ) \rightarrow 1$ (when $\vectq \rightarrow 0$), namely:
\begin{equation*}
  \lim_{\vectq \rightarrow 0} I(\vectq ) = \ensavg{\alpha}{\left| F_\alpha(0 ) \right| ^2} + \left( N-1 \right) \left| \ensavg{\alpha}{F_\alpha(0 )} \right|^2 \; .
\end{equation*}

%%%%%%%%%%%%%%%%%%%%%%%%%%%%%%%%%%%%%
\section{Distorted Wave Born Approximation} 
In this section, one proceeds along similar lines as in the formal treatment of section \ref{sec:formal}. This time however, the full Hamiltonian is written as $H_2 = H_1 + V_2 = H_0 +V_1 + V_2$, where $H_0$ will again refer to the free Hamiltonian. In the distorted wave Born approximation (DWBA), one performs a perturbative expansion around the solutions of the Hamiltonian $H_1$, which are assumed to be known:
\begin{align*}
  H_1\Psi^\pm_{1\alpha} &= E_\alpha\Psi^\pm_{1\alpha} \\
  \Psi^\pm_{1\alpha} &= \Psi_{0\alpha} + G^\pm_1 V_1 \Psi_{0\alpha} \; ,
\end{align*}
where the Green operators are defined to be:
\begin{align*}
  G^\pm_1 &\equiv (E-H_1\pm i\epsilon) ^{-1} \nonumber \\
  G^\pm_2 &\equiv (E-H_2\pm i\epsilon) ^{-1} \; .
\end{align*}

The $T$--matrix element for scattering between the asymptotic states $\Psi_{0\alpha}$ and $\Psi_{0\beta}$ (note that these asymptotic states refer to the free Hamiltonian $H_0$), is:
\begin{align*}
  T^+_{\alpha\beta} &= \braket{\Psi_{0\beta}}{V_1+V_2}{\Psi^+_\alpha} \nonumber \\
  & = \braket{\Psi_{0\beta}}{V_1+V_2}{\Psi_{0\alpha} + G^+_1V_1\Psi_{0\alpha} + G^+_2V_2\Psi^+_{1\alpha}} \nonumber \\
  & = \braket{\Psi_{0\beta}}{V_1}{\Psi^+_{1\alpha}} + \braket{\Psi_{0\beta}}{(V_1G^+_1 + 1)V_2}{\Psi^+_\alpha} \\
  & = \braket{\Psi_{0\beta}}{V_1}{\Psi^+_{1\alpha}} + \braket{\Psi^-_{1\beta}}{V_2}{\Psi^+_\alpha} \nonumber \\
  & = \braket{\Psi_{0\beta}}{V_1}{\Psi^+_{1\alpha}} + \braket{\Psi^-_{1\beta}}{T_2}{\Psi^+_{1\alpha}} \; ,
\end{align*}
with $T_2 = V_2 + V_2G^+_2V_2$. By approximating this last term using $T_2 \simeq V_2$, one arrives at the distorted wave Born approximation:
\begin{equation}
  \label{eq:tdwba}
   T^+_{\alpha\beta} \simeq \braket{\Psi_{0\beta}}{V_1}{\Psi^+_{1\alpha}} + \braket{\Psi^-_{1\beta}}{V_2}{\Psi^+_{1\alpha}} \; .
\end{equation}

%%%%%%%%%%%%%%%%%%%%%%%%%%%%%%%%%%%%%
\subsection{Multilayer systems}
In multilayer systems, the first term of \refeq{tdwba} denotes the specular part of the reflection, while the second term is responsible for the off-specular scattering. This off-specular part is caused by deviations from the perfectly smooth layered system, as e.g. interface roughnesses or included nanoparticles. In here only the case of nanoparticles will be treated.

In the conventions where $H=-\Delta + V$, the potential splits into two parts $V_1$ and $V_2$, where only the second part is treated as a perturbation:
\begin{align*}
  V_1 & = k_0^2\left( 1-n_0^2(\vect{r})\right)  \\
  V_2 & = \sum_i k_0^2\left( n_0^2(\vect{R}^i) - n_i^2 \right) S^i(\vect{r}) \otimes \delta(\vect{r}-\vect{R}^i) \; ,
\end{align*}
where $n_0(\vect{r})$ denotes the refractive index of the unperturbed system (which, in case of a multilayer system, will only depend on its $z$--coordinate) and $n_i$ is the refractive index of the nanoparticle with shape function $S^i$ and position $\vect{R}^i$.

For nanoparticles in a specific layer $j$, i.e. $V_2\neq0$ only in layer $j$, one only needs the unperturbed solutions in layer $j$:
\begin{align*}
  \braketnoop{\vect{r}}{\Psi^+_{1k_i}} &= (2\pi)^{-3/2}\left[ R_j(\vect{k}_i) e^{i \vect{k}_{j,R}(\vect{k}_i)\cdot\vect{r}} + T_j(\vect{k}_i) e^{i \vect{k}_{j,T}(\vect{k}_i)\cdot\vect{r}} \right] \\
  \braketnoop{\Psi^-_{1k_f}}{\vect{r}} &= (2\pi)^{-3/2}\left[ R_j(-\vect{k}_f) e^{i \vect{k}_{j,R}(-\vect{k}_f)\cdot\vect{r}} + T_j(-\vect{k}_f) e^{i \vect{k}_{j,T}(-\vect{k}_f)\cdot\vectr} \right] \; .
\end{align*}

The off--specular contribution to the scattering amplitude then becomes:
\begin{align*}
  f(\theta, \phi) &= -\int d^3\vectr \frac{V_2(\vectr)}{4\pi} \biggl[ T_iT_fe^{i(\vectk_{j,i}-\vectk_{j,f})\cdot\vectr} + R_iT_fe^{i(\vectkt_{j,i}-\vectk_{j,f})\cdot\vectr} \\
   & + T_iR_fe^{i(\vectk_{j,i}-\vectkt_{j,f})\cdot\vectr} + R_iR_fe^{i(\vectkt_{j,i}-\vectkt_{j,f})\cdot\vectr} \biggr] \; ,
\end{align*}
where the following shorthand notations were used:
\begin{align*}
  T_i &\equiv  T_j(\vect{k}_i) & R_i &\equiv  R_j(\vect{k}_i)  \\
  T_f &\equiv  T_j(-\vect{k}_f) & R_f &\equiv  R_j(-\vect{k}_f) \\
  \vectk_{j,i} &\equiv \vectk_{j,T}(\vectk_i) & \vectkt_{j,i} &\equiv \vectk_{j,R}(\vectk_i)  \\
  \vectk_{j,f} &\equiv -\vectk_{j,T}(-\vectk_f) & \vectkt_{j,f} &\equiv -\vectk_{j,R}(-\vectk_f) \; .
\end{align*}

From this expression, one sees that the scattering amplitude consists of a weighted sum of Fourier transforms of the potential $V_2$. Using
\begin{equation*}
  V_2(\vectr) = \sum_i 4\pi \rho_{s,rel,i} S^i(\vectr) \otimes \delta(\vectr - \vect{R}^i) \; ,
\end{equation*}
with $\rho_{s,rel,i}\equiv  k_0^2\left( n_0^2(\vect{R}^i) - n_i^2 \right)/4\pi$, the scattering amplitude becomes
\begin{equation*}
  f(\theta, \phi) = -\sum_i  \rho_{s,rel,i} \curlf^i_{\text{DWBA}}(\vectk_{j,i},\vectk_{j,f},\vect{R}^i_z)e^{i(\vectk_{j,i\parallel}-\vectk_{j,f\parallel})\cdot \vect{R}^{i\parallel} } \; ,
\end{equation*}
with
\begin{align*}
  \curlf^i_\text{DWBA}(\vectk_i,\vectk_f,R_z) & \equiv T_iT_fF^i(\vectk_i-\vectk_f)e^{i(k_{iz}-k_{fz})R_z} + R_iT_fF^i(\vectkt_i-\vectk_f)e^{i(-k_{iz}-k_{fz})R_z} \\
  & + T_iR_fF^i(\vectk_i-\vectkt_f)e^{i(k_{iz}+k_{fz})R_z} + R_iR_fF^i(\vectkt_i-\vectkt_f)e^{i(-k_{iz}+k_{fz})R_z} \; ,
\end{align*}

With this last expression, the same techniques as demonstrated in section \ref{sec:ba} can be applied, leading to the following expression for the expectation value of the scattering cross--section:
\begin{align*}
  & \left\langle \frac{d\sigma}{d\Omega}(\vectk_i,\vectk_f) \right\rangle_{\text{Off--specular}}  \\
  & = \sum_\alpha p_\alpha \left\lvert \curlf_\alpha(\vectk_{j,i},\vectk_{j,f}, R_{\alpha,z})\right\rvert ^2 + \frac{\rho_S}{S}\sum_{\alpha,\beta} p_\alpha p_\beta \curlf_\alpha (\vectk_{j,i},\vectk_{j,f}, R_{\alpha,z})\curlf_\beta^*(\vectk_{j,i},\vectk_{j,f}, R_{\beta,z}) \\
  & \times \iint_S d^2\vect{R}_\alpha^\parallel d^2\vect{R}_\beta^\parallel \ppcf{\alpha}{\beta}{R^\parallel} \exp \left[ i\vect{q}_{j\parallel}\cdot (\vect{R}_\alpha^\parallel - \vect{R}_\beta^\parallel ) \right] \; .
\end{align*}

The main differences with respect to the cross---section in the Born approximation are:
\begin{enumerate}
  \item The particle form factor now consists of a more complex expression and now depends on both incoming and outgoing wavevectors and also on the $z$--coordinate of the particle;
  \item Since the $z$--coordinate of the particles is implicitly included in its formfactor, the position integrals only run over $x$-- and $y$--coordinates and the volume and density gets replaced with the surface area and surface density respectively.
\end{enumerate}

