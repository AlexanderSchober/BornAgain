\newpage
\section{Quick start} \SecLabel{QuickStart}

This section shortly describes how to build \BornAgain\ from source and run first
simulation. More details about software architecture and installation
procedure are given in \SecRef{SoftwareArchitecture} and \SecRef{InstallationProcedure}. \\

\noindent
{\bf Step I: $~$ installing third party libraries}
\begin{itemize}
\item boost library ($\geq 1.48$)
\item GNU scientific library ($\geq 1.15$)
\item fftw3 library ($\geq 3.3.1$)
\item Eigen3 library ($\geq 3.1.0$), optional
\item ROOT framework ($\geq 5.34.00$), optional
\end{itemize}
\vspace*{2mm}


\noindent
{\bf Step II: $~$ getting the source}
\begin{lstlisting}[language=bash, style=commandline]
git clone git://apps.jcns.fz-juelich.de/BornAgain.git 
\end{lstlisting}
\vspace*{3mm}


\noindent
{\bf Step III: $~$ building the source}
\begin{lstlisting}[language=shell, style=commandline]
mkdir <build_dir>; cd <build_dir>;
cmake <source_dir> -DCMAKE_INSTALL_PREFFIX=<install_dir>
make
make check
make install
\end{lstlisting}
\vspace*{3mm}


\noindent
{\bf Step IV: $~$ running example}
\begin{lstlisting}[language=shell, style=commandline]
cd <install_dir>/Examples/python/ex001_CylindersAndPrisms
python CylindersAndPrisms.py
\end{lstlisting}


%Requirements

%Hardware
%BornAgain is known to work on following platforms:
%Linux (x86, amd64)
%MacOS X (x86)

%Software
%GCC 4.1.2 or above   C/C++ compiler
%or
%clang
%gcc 4.1.2 or above, clang
