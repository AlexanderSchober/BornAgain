%%%%%%%%%%%%%%%%%%%%%%%%%%%%%%%%%%%%%%%%%%%%%%%%%%%%%%%%%%%%%%%%%%%%%%%%%%%%%%%%
%%
%%   BornAgain User Manual
%%
%%   homepage:   http://www.bornagainproject.org
%%
%%   copyright:  Forschungszentrum Jülich GmbH 2015
%%
%%   license:    Creative Commons CC-BY-SA
%%   
%%   authors:    Scientific Computing Group at MLZ Garching
%%               C. Durniak, M. Ganeva, G. Pospelov, W. Van Herck, J. Wuttke
%%
%%%%%%%%%%%%%%%%%%%%%%%%%%%%%%%%%%%%%%%%%%%%%%%%%%%%%%%%%%%%%%%%%%%%%%%%%%%%%%%%


\chapter{Foundations of GISAS}  \SecLabel{ScatTheory}


%%%%%%%%%%%%%%%%%%%%%%%%%%%%%%%%%%%%%%%%%%%%%%%%%%%%%%%%%%%%%%%%%%%%%%%%%%%%%%%%
\section{Wave propagation}
%%%%%%%%%%%%%%%%%%%%%%%%%%%%%%%%%%%%%%%%%%%%%%%%%%%%%%%%%%%%%%%%%%%%%%%%%%%%%%%%
\index{Wave propagation|(}

%===============================================================================
\subsection{Helmholtz and Schr\"odinger equation}
%===============================================================================

X-ray and neutron propagation are both governed by
the same stationary wave equation,
the \textit{Helmholtz equation}
\index{Helmholtz equation}
\index{Wave equation}
\begin{equation}\label{eq:Helmholtz}
  \left(\Nabla^2+K^2n(\vect{r})^2\right)\Psi(\vect{r}),
\end{equation}
where $K=2\pi/\lambda_0$ is the vacuum wave number,
and $n$ is the refractive index.
\index{Refractive index}
In this equation,
the time dependence ${\rm e}^{i\omega t}$ of the wave field
does not appear explicitly.
In GISAS, we are only interested in elastic scattering.
Therefore, the wave frequency~$\omega$ is considered constant.

Implicitly, $K$ depends on~$\omega$ through a \textit{dispersion relation}.
\index{dispersion relation}
This relation is fundamentally different for
X-rays (where it is linear, $\omega\propto K$)
and for thermal neutrons (where it is quadratic, $\omega\propto K^2$).
However, for stationary problems
this frequency dependence of~$K$ does not matter.
Therefore we can use $K$ instead of~$\omega$ as a constant
that characterizes the incoming radiation.
In this way, scalar X-ray and neutron scattering can be described by exactly
the same wave equation~\refeq{Helmholtz}.

In the following,
the Helmholtz equation 
shall be solved in first-order perturbation theory,
using the fact that the refractive index~$n$
is close to~1 for both X-rays and neutrons.
For this purpose,
we rewrite \refeq{Helmholtz} as a stationary Schr\"odinger equation
\index{Schrodinger equation@Schr\"odinger equation}
\begin{equation}
  H\Psi = E\Psi
\end{equation}
with
\begin{equation*}
  H:=H_0+V,\quad
  H_0:=\Nabla^2,\quad
  V:=-K^2(1-n(\vect{r}^2),\quad
  E:=K^2.
\end{equation*}

Some GISAS experiments are performed with polarized radiation
or/and involve polarization-dependent interactions.
Therefore we need also to consider
the propagation of polarized X-rays and neutrons.
Electromagnetic field vectors and neutron spinors
are fundamentally different mathematical objects
Nevertheless, for our purpose they can be
mapped onto a uniform .... <to elaborate> ....
so that one and the same formalism can be applied
to polarized GISAXS and GISANS.
This shall be derived in the following.

%===============================================================================
\subsection{X-ray propagation}
%===============================================================================
\index{Wave propagation!X-rays}
\index{X-rays!wave propagation}

\ldots to come

%===============================================================================
\subsection{Neutron propagation}
%===============================================================================
\index{Wave propagation!neutrons}
\index{neutrons!wave propagation}

In a GISANS experiment,
we are only interested in small scattering vectors~$\vect{q}$,
and therefore do not register ordinary Bragg scattering
by crystal lattices or other correlations at atomic level;
at most, Bragg scattering must be accounted for as a loss channels.
We therefore describe the neutron-sample interaction
in continuum approximation by a refractive index~$n$ with
\begin{equation}
  n^2=1-\frac{4\pi}{K^2}\rho_s
\end{equation}
where $\rho_s$ is an effective scattering length density
\cite{Lax51,Sea89}.

%TODO: how is absorption taken into account?

\index{Wave propagation|)}


%%%%%%%%%%%%%%%%%%%%%%%%%%%%%%%%%%%%%%%%%%%%%%%%%%%%%%%%%%%%%%%%%%%%%%%%%%%%%%%%
\section{Scattering in Born approximation}
%%%%%%%%%%%%%%%%%%%%%%%%%%%%%%%%%%%%%%%%%%%%%%%%%%%%%%%%%%%%%%%%%%%%%%%%%%%%%%%%


%===============================================================================
\subsection{Green operators and the $T$-matrix}
  \label{sec:BornT}
%===============================================================================

For a particle, governed by the Schr\"odinger equation with Hamiltonian $H = H_0 + V$, the time-independent scattering theory formally consists of solving the eigenvalue equations:

\begin{equation*}
  H\Psi_\alpha = E_\alpha\Psi_\alpha,
\end{equation*}
with $E$ the scalar energy eigenvalue of the eigenstate $\Psi(E)$.\\
If the solutions of the free (or unperturbed) Hamiltonian $H_0$ are known:
\begin{equation*}
  H_0\Psi_{0\alpha} = E_\alpha\Psi_{0\alpha},
\end{equation*}
one can write the solutions of the full Hamiltonian in terms of these asymptotic states and Green operators:

\begin{align*}
  \Psi^\pm_\alpha &= \Psi_{0\alpha} + G^\pm_0 V \Psi^\pm_\alpha  \\
  & = \Psi_{0\alpha} + G^\pm V  \Psi_{0\alpha},
\end{align*}
where the Green operators are defined as:
\begin{align*}
  G^\pm_0(E) &= (E-H_0\pm i\epsilon )^{-1}  \\
  G^\pm (E) &= (E-H\pm i\epsilon )^{-1}.
\end{align*}
In these equations, the upper index or sign refers to the state corresponding with the free state $\Psi_{0\alpha}$ at time $t\rightarrow - \infty$ (and vice-versa for the lower sign). Since the solutions of the eigenvalue equations, both for the unperturbed as for the full Hamiltonian, are dependent on the energy eigenvalue $E$, the index $\alpha$ is assumed to include this value (and possibly other quantum numbers).

The transition amplitude between two asymptotic states is given by the $S$-matrix elements, defined as:
\begin{align}
  S_{\alpha\beta} &\equiv \braket{\Psi_{0\beta}}{S}{\Psi_{0\alpha}} \nonumber\\
  & \equiv \braketnoop{\Psi_\beta^-}{\Psi_\alpha^+}.
  \label{<++>}
\end{align}

The $S$-matrix can be decomposed into a delta function, representing the absence of scattering, and a $T$-matrix that encodes the scattering part, caused by the potential $V$:
\begin{equation*}
  S_{\alpha\beta} = \delta (E_\alpha - E_\beta)\delta_{\alpha\beta} - 2\pi i \delta (E_\alpha - E_\beta) T^\pm_{\alpha\beta},
\end{equation*}
with
\begin{align*}
  T^+_{\alpha\beta} & = \braket{\Psi_{0\beta}}{V}{\Psi^+_\alpha} \\
  T^-_{\alpha\beta} & = \braket{\Psi^-_\beta}{V}{\Psi_{0\alpha}}.
\end{align*}
On the energy shell $E_\alpha = E_\beta$, one has $T^+_{\alpha\beta}= T^-_{\alpha\beta}$, so that both formulations are equivalent.

By expanding the eigenstates $\Psi^\pm_\alpha$ in these equations, the $T$-matrix elements (on-shell) can be expressed as:
\begin{equation*}
  T^\pm_{\alpha\beta} = V + VG^+ V.
\end{equation*}


%===============================================================================
\subsection{Momentum representation and the scattering cross-section}
%===============================================================================

The previous general formulas can also be presented in a momentum (and position) eigenbasis, defined by:
\begin{align*}
  \oper{P}\ket{\vect{k}} & = \hbar \vect{k} \ket{\vect{k}} \\
  \braketnoop{\vect{k'}}{\vect{k}} & = \delta(\vect{k'}-\vect{k}) \\
  1 &= \int d^3\vect{k} \ket{\vect{k}} \bra{\vect{k}} \\
  1 &= \int d^3\vect{r} \ket{\vect{r}} \bra{\vect{r}} \\
  \braketnoop{\vect{r}}{\vect{k}}
  &= (2\pi)^{-3/2} \exp (i\vect{k}\cdot\vect{r}),
\end{align*}
where the normalization in the last equation follows from the other definitions.

The wavefunction that evolves from a momentum eigenstate $\ket{\vect{k}_i}$ can then be written as:
\begin{equation*}
  \braketnoop{\vect{r}}{\Psi^+} = \braketnoop{\vect{r}}{\vect{k}_i} + \braket{\vect{r}}{G_0^+T}{\vect{k}_i},
\end{equation*}
which in the far-field limit becomes:
\begin{align*}
  \braketnoop{\vect{r}}{\Psi^+}
  &= (2\pi)^{-3/2}\left[ e^{i\vect{k}_i\cdot\vect{r}}
     - 2\pi^2\frac{e^{ik_f r}}{r} \braket{\vect{k}_f}{T}{\vect{k}_i} \right]  \\
  &\equiv (2\pi)^{-3/2}\left[ e^{i\vect{k}_i\cdot\vect{r}} + f(\theta,\phi)\frac{e^{ik_f r}}{r} \right],
\end{align*}
where $r,\theta,\phi$ are the polar coordinates of $\vect{r}$,
and the scattering amplitude was defined as
\begin{equation*}
  f(\theta, \phi) := -2\pi^2\braket{\vect{k}_f}{T}{\vect{k}_i}.
\end{equation*}

The amount of particles per unit time that are scattered in a small solid angle $d\Omega$ in direction $\vect{k}_f$ will then be (still in the far-field limit):
\begin{equation*}
  dI_{scat} = J_0\lvert f(\theta,\phi)\rvert^2d\Omega,
\end{equation*}
where $J_0$ denotes the incident flux density. The scattering cross-section is defined as:
\begin{equation*}
  \frac{d\sigma}{d\Omega}\equiv \frac{dI_{scat}}{J_0 d\Omega} = \lvert f(\theta,\phi) \rvert^2.
\end{equation*}

Consider a scattering volume $V$, containing $N$ scattering centers with shape functions $S^i(\vect{r})$, positions $\vect{R}^i$ and scattering length density $\rho_s$ (relative to the ambient material).

In the Born approximation ($T\simeq V$), the scattering amplitude is
\begin{align*}
  f(\theta, \phi) & = -8\pi^3 \braket{\vect{k_f}}{\rho_s(\vect{r})}{\vect{k_i}} \nonumber \\
  & = -\int d^3\vect{r} e^{i\vect{q}\cdot\vect{r}} \rho_s(\vect{r}),
\end{align*}
where $\vect{q}:=\vect{k}_i-\vect{k}_f$ denotes the scattering wavevector.

The differential cross-section (per scattering center) is then given by:
\begin{equation}\label{eq:xsec1}
  \frac{d\sigma}{d\Omega}(\vect{q}) = \frac{1}{N}\left\lvert \int_V \rho_s(\vect{r}) e^{i\vect{q}\cdot\vect{r}} d^3\vect{r} \right\rvert ^2.
\end{equation}



%%%%%%%%%%%%%%%%%%%%%%%%%%%%%%%%%%%%%%%%%%%%%%%%%%%%%%%%%%%%%%%%%%%%%%%%%%%%%%%%
\section{Distorted Wave Born Approximation} 
%%%%%%%%%%%%%%%%%%%%%%%%%%%%%%%%%%%%%%%%%%%%%%%%%%%%%%%%%%%%%%%%%%%%%%%%%%%%%%%%
\index{Distorted wave Born approximation}
\index{DWBA|see {Distorted wave Born approximation}}

In Born approximation,
the incoming radiation is treated as a plane wave
with wavevector~$\vect{k}_i$.
The scattered spherical wave is considered
at a particular remote detector position,
and therefore can be approximated as an outgoing plane wave
with wavevector~$\vect{k}_f$.
This is adequate for almost all neutron scattering experiments ---
but not if plane-wave propagation is substantially distorted
by reflection and refraction.
Which is most notably the case
for grazing-incidence scattering from multilayer samples.
While scattering can still be treated as a weak perturbation,
reflection and refraction distort the incoming and outgoing wave field
in zeroeth order,
and need to be fully accounted for.
Doing this, the Born approximation is naturally extended into
the distorted wave Born approximation (DWBA).

To start, we extend the formalism of section \ref{sec:BornT}.
The full Hamiltonian is now written as $H_2 = H_1 + V_2 = H_0 +V_1 + V_2$,
where $H_0$ will again refer to the free Hamiltonian.
In DWBA,
one performs a perturbative expansion
around the solutions of the Hamiltonian $H_1$,
which are assumed to be known:
\begin{align*}
  H_1\Psi^\pm_{1\alpha} &= E_\alpha\Psi^\pm_{1\alpha} \\
  \Psi^\pm_{1\alpha} &= \Psi_{0\alpha} + G^\pm_1 V_1 \Psi_{0\alpha},
\end{align*}
where the Green operators are defined to be:
\begin{align*}
  G^\pm_1 &\equiv (E-H_1\pm i\epsilon) ^{-1} \nonumber \\
  G^\pm_2 &\equiv (E-H_2\pm i\epsilon) ^{-1}.
\end{align*}

The $T$-matrix element for scattering between the asymptotic states $\Psi_{0\alpha}$ and $\Psi_{0\beta}$ (note that these asymptotic states refer to the free Hamiltonian $H_0$), is:
\begin{align*}
  T^+_{\alpha\beta} &= \braket{\Psi_{0\beta}}{V_1+V_2}{\Psi^+_\alpha} \nonumber \\
  & = \braket{\Psi_{0\beta}}{V_1+V_2}{\Psi_{0\alpha} + G^+_1V_1\Psi_{0\alpha} + G^+_2V_2\Psi^+_{1\alpha}} \nonumber \\
  & = \braket{\Psi_{0\beta}}{V_1}{\Psi^+_{1\alpha}} + \braket{\Psi_{0\beta}}{(V_1G^+_1 + 1)V_2}{\Psi^+_\alpha} \\
  & = \braket{\Psi_{0\beta}}{V_1}{\Psi^+_{1\alpha}} + \braket{\Psi^-_{1\beta}}{V_2}{\Psi^+_\alpha} \nonumber \\
  & = \braket{\Psi_{0\beta}}{V_1}{\Psi^+_{1\alpha}} + \braket{\Psi^-_{1\beta}}{T_2}{\Psi^+_{1\alpha}},
\end{align*}
with $T_2 = V_2 + V_2G^+_2V_2$. By approximating this last term using $T_2 \simeq V_2$, one arrives at the distorted wave Born approximation:
\begin{equation}
  \label{eq:tdwba}
   T^+_{\alpha\beta} \simeq \braket{\Psi_{0\beta}}{V_1}{\Psi^+_{1\alpha}} + \braket{\Psi^-_{1\beta}}{V_2}{\Psi^+_{1\alpha}}.
\end{equation}


%===============================================================================
\subsection{Multilayer systems}
%===============================================================================

In multilayer systems, the first term of \refeq{tdwba} denotes the specular part of the reflection, while the second term is responsible for the off-specular scattering. This off-specular part is caused by deviations from the perfectly smooth layered system, as e.g. interface roughnesses or included nanoparticles. In here only the case of nanoparticles will be treated.

In the conventions where $H=-\Delta + V$, the potential splits into two parts $V_1$ and $V_2$, where only the second part is treated as a perturbation:
\begin{align*}
  V_1 & = K^2\left( 1-n_0^2(\vect{r})\right)  \\
  V_2 & = \sum_i K^2\left( n_0^2(\vect{R}^i) - n_i^2 \right) S^i(\vect{r}) \otimes \delta(\vect{r}-\vect{R}^i),
\end{align*}
where $n_0(\vect{r})$ denotes the refractive index of the unperturbed system (which, in case of a multilayer system, will only depend on its $z$-coordinate) and $n_i$ is the refractive index of the nanoparticle with shape function $S^i$ and position $\vect{R}^i$.

For nanoparticles in a specific layer $j$, i.e. $V_2\neq0$ only in layer $j$, one only needs the unperturbed solutions in layer $j$:
\begin{align*}
  \braketnoop{\vect{r}}{\Psi^+_{1k_i}} &= (2\pi)^{-3/2}\left[ R_j(\vect{k}_i) e^{i \vect{k}_{j,R}(\vect{k}_i)\cdot\vect{r}} + T_j(\vect{k}_i) e^{i \vect{k}_{j,T}(\vect{k}_i)\cdot\vect{r}} \right] \\
  \braketnoop{\Psi^-_{1k_f}}{\vect{r}} &= (2\pi)^{-3/2}\left[ R_j(-\vect{k}_f) e^{i \vect{k}_{j,R}(-\vect{k}_f)\cdot\vect{r}} + T_j(-\vect{k}_f) e^{i \vect{k}_{j,T}(-\vect{k}_f)\cdot\vectr} \right].
\end{align*}

The off-specular contribution to the scattering amplitude then becomes:
\begin{align*}
  f(\theta, \phi) &= -\int d^3\vectr \frac{V_2(\vectr)}{4\pi} \biggl[ T_iT_fe^{i(\vectk_{j,i}-\vectk_{j,f})\cdot\vectr} + R_iT_fe^{i(\vectkt_{j,i}-\vectk_{j,f})\cdot\vectr} \\
   & + T_iR_fe^{i(\vectk_{j,i}-\vectkt_{j,f})\cdot\vectr} + R_iR_fe^{i(\vectkt_{j,i}-\vectkt_{j,f})\cdot\vectr} \biggr],
\end{align*}
where the following shorthand notations were used:
\begin{align*}
  T_i &\equiv  T_j(\vect{k}_i) & R_i &\equiv  R_j(\vect{k}_i)  \\
  T_f &\equiv  T_j(-\vect{k}_f) & R_f &\equiv  R_j(-\vect{k}_f) \\
  \vectk_{j,i} &\equiv \vectk_{j,T}(\vectk_i) & \vectkt_{j,i} &\equiv \vectk_{j,R}(\vectk_i)  \\
  \vectk_{j,f} &\equiv -\vectk_{j,T}(-\vectk_f) & \vectkt_{j,f} &\equiv -\vectk_{j,R}(-\vectk_f).
\end{align*}

From this expression, one sees that the scattering amplitude consists of a weighted sum of Fourier transforms of the potential $V_2$. Using
\begin{equation*}
  V_2(\vectr) = \sum_i 4\pi \rho_{s,rel,i} S^i(\vectr) \otimes \delta(\vectr - \vect{R}^i),
\end{equation*}
with $\rho_{s,rel,i}\equiv  K^2\left( n_0^2(\vect{R}^i) - n_i^2 \right)/4\pi$, the scattering amplitude becomes
\begin{equation*}
  f(\theta, \phi) = -\sum_i  \rho_{s,rel,i} \curlf^i_{\text{DWBA}}(\vectk_{j,i},\vectk_{j,f},\vect{R}^i_z)e^{i(\vectk_{j,i\parallel}-\vectk_{j,f\parallel})\cdot \vect{R}^{i\parallel} },
\end{equation*}
with
\begin{align*}
  \curlf^i_\text{DWBA}(\vectk_i,\vectk_f,R_z) & \equiv T_iT_fF^i(\vectk_i-\vectk_f)e^{i(k_{iz}-k_{fz})R_z} + R_iT_fF^i(\vectkt_i-\vectk_f)e^{i(-k_{iz}-k_{fz})R_z} \\
  & + T_iR_fF^i(\vectk_i-\vectkt_f)e^{i(k_{iz}+k_{fz})R_z} + R_iR_fF^i(\vectkt_i-\vectkt_f)e^{i(-k_{iz}+k_{fz})R_z},
\end{align*}

TO MERGE IN:
In the DWBA, the form factor of a particle in a multilayer system is given by
\begin{align}
F_{\rm{DWBA}} (\vect{k}_i,\vect{k}_f, r_z) & = T_i T_f F_{\rm{BA}} (\vect{k}_i-\vect{k}_f) e^{i (k_{i,z}-k_{f,z}) r_z} + R_i T_f F_{\rm{BA}}(\vect{\widetilde{k}}_i-\vect{k}_f) e^{i(-k_{i,z}-k_{f,z})r_z}
 \nonumber \\
  &+ T_i R_f F_{\rm{BA}}(\vect{k}_i-\vect{\widetilde{k}}_f)e^{i(k_{i,z}+k_{f,z})r_z} + R_iR_fF_{\rm{BA}} (\vect{\widetilde{k}}_i-\vect{\widetilde{k}}_f)e^{i(-k_{i,z}+k_{f,z})r_z}, \label{eq:dwbageneral}
\end{align}
where $F_{\rm{BA}}$ is the expression of the form factor in the Born approximation, $r_z$ is the $z$-coordinate of the particle's position (measured from the bottom of the particle), $\vect{k}_i=(k_{i,x}, k_{i,y}, k_{i,z})$ $\vect{k}_f=(k_{f,x}, k_{f,y}, k_{f,z})$ are the incident and scattered wave vectors in air, respectively \cite{RaSS95}. With a tilde (\~{}), these wavevectors components are evaluated in the multilayer system (the refractive indices of the different constituting materials have to be taken into account). 
$T_i$, $T_f$, $R_i$, $R_f$ are the transmission and reflection coefficients for the incident wave (index $i$) or the scattered one (index $f$). These coefficients can be calculated using the Parratt formalism \cite{Par54} or the matrix method \cite{BoWo99}. $\vect{k}_i-\vect{k}_f$ is equal to the scattering vector $\vect{q}$ and the $z$-axis is pointing upwards.\\

With this last expression, the same techniques as demonstrated in section \ref{sec:ba} can be applied, leading to the following expression for the expectation value of the scattering cross-section:
\begin{align*}
  & \left\langle \frac{d\sigma}{d\Omega}(\vectk_i,\vectk_f) \right\rangle_{\text{Off-specular}}  \\
  & = \sum_\alpha p_\alpha \left\lvert \curlf_\alpha(\vectk_{j,i},\vectk_{j,f}, R_{\alpha,z})\right\rvert ^2 + \frac{\rho_S}{S}\sum_{\alpha,\beta} p_\alpha p_\beta \curlf_\alpha (\vectk_{j,i},\vectk_{j,f}, R_{\alpha,z})\curlf_\beta^*(\vectk_{j,i},\vectk_{j,f}, R_{\beta,z}) \\
  & \times \iint_S d^2\vect{R}_\alpha^\parallel d^2\vect{R}_\beta^\parallel \ppcf{\alpha}{\beta}{R^\parallel} \exp \left[ i\vect{q}_{j\parallel}\cdot (\vect{R}_\alpha^\parallel - \vect{R}_\beta^\parallel ) \right].
\end{align*}

The main differences with respect to the cross-section in the Born approximation are:
\begin{enumerate}
  \item The particle form factor now consists of a more complex expression and now depends on both incoming and outgoing wavevectors and also on the $z$-coordinate of the particle;
  \item Since the $z$-coordinate of the particles is implicitly included in its formfactor, the position integrals only run over $x$- and $y$-coordinates and the volume and density gets replaced with the surface area and surface density respectively.
\end{enumerate}


