%%%%%%%%%%%%%%%%%%%%%%%%%%%%%%%%%%%%%%%%%%%%%%%%%%%%%%%%%%%%%%%%%%%%%%%%%%%%%%%%
%%
%%   BornAgain User Manual
%%
%%   homepage:   http://www.bornagainproject.org
%%
%%   copyright:  Forschungszentrum Jülich GmbH 2015
%%
%%   license:    Creative Commons CC-BY-SA
%%   
%%   authors:    Scientific Computing Group at MLZ Garching
%%               C. Durniak, M. Ganeva, G. Pospelov, W. Van Herck, J. Wuttke
%%
%%%%%%%%%%%%%%%%%%%%%%%%%%%%%%%%%%%%%%%%%%%%%%%%%%%%%%%%%%%%%%%%%%%%%%%%%%%%%%%%

\let\v=\vect
\def\r{\v{r}}
\def\k{\v{k}}
\def\rD{\r_\text{D}}
\def\rS{\r_\text{S}}
\def\kD{\k_\text{D}}
\def\Sample{\mathcal{M}}
\def\Sphere{\mathcal{S}}

\chapter{Foundations of GISAS}  \SecLabel{ScatTheory}

\def\nz{\overline{n^2}}
\def\nzj{\overline{n_j^2}}

In this chapter,
we present the theory
of grazing-incidence small-angle scattering (GISAS)
that underlies the simulation algorithm implemented in BornAgain.
We specifically consider the propagation and scattering
of neutrons with no polarization-dependent interactions,
so that we can base our exposition on a scalar wave equation.
The notationally more involved
vectorial theory needed for X-rays and for polarized neutrons
is adjourned to the next chapter.

Our exposition is largely self-contained,
except for the initial passage from the microscopic
to the macroscopic Schrödinger equation,
which we outline only briefly.
This manual is decidedly not the place for renarrating
the historic development of GISAS theory.
For ample references to the original literature,
we refer to the excellent review \cite{ReLL09}.


%%%%%%%%%%%%%%%%%%%%%%%%%%%%%%%%%%%%%%%%%%%%%%%%%%%%%%%%%%%%%%%%%%%%%%%%%%%%%%%%
\section{Coherent neutron propagation}\label{Swave}
%%%%%%%%%%%%%%%%%%%%%%%%%%%%%%%%%%%%%%%%%%%%%%%%%%%%%%%%%%%%%%%%%%%%%%%%%%%%%%%%
\index{Wave propagation!neutrons|(}%
\index{Neutrons!wave propagation|(}%

\index{Schrodinger@Schrödinger equation!microscopic}%
The scalar wave function $\psi(\r,t)$ of a slow neutron
is governed by the microscopic Schrödinger equation
\begin{equation}\label{ESchrodi}
  i\hbar\partial_t \psi(\r,t)
  = \left\{-\frac{\hbar^2}{2m}\Nabla^2+V(\r)\right\} \psi(\r,t).
\end{equation}
By assuming a time-independent potential $V(\r)$,
we have excluded inelastic scattering.
Therefore we need only to consider monochromatic waves
with given frequency~$\omega$.
In consequence, we have a stationary wave function
\begin{equation}\label{Estationarywave}
  \psi(\r,t) = \psi(\r){\rm e}^{-i\omega t}.
\end{equation}
The minus sign in the exponent of the phase factor
is an inevitable consequence of the standard form of the Schrödinger equation,
and is therefore called the \textit{quantum-mechanical sign convention}.
\index{Wave propagation!sign convention}%
\index{Sign convention}%
The opposite choice of a phase factor ${\rm e}^{+i\omega t}$ is 
called the \textit{crystallographic sign convention},
and is used in much of the literature on X-ray scattering,
including some important texts on GISAXS (e.~g.\ \cite{ReLL09}).
Since BornAgain is covering both X-ray and neutron scattering,
the obvious choice for us was to adopt the quantum-mechanical convention
throughout the source code as well as in this manual.

Inserting (\ref{Estationarywave}) in (\ref{ESchrodi}),
we obtain the stationary Schrödinger equation
\begin{equation}\label{EstatSchrodi}
  \left\{-\frac{\hbar^2}{2m}\Nabla^2+V(\r)-\hbar\omega\right\} \psi(\r) = 0.
\end{equation}
\index{Potential|see {Optical potential}}%
\index{Optical potential!nuclear (microscopic)}%
The \textit{nuclear} (or \textit{microscopic})
\textit{optical potential} $V(\r)$,
in a somewhat ``naive conception'' \cite[p.~7]{Sea89},
consists of a sum of delta functions,
representing Fermi's ``pseudopotential''.
\index{Fermi's pseudopotential}%
The superposition of the incident wave with the scattered waves
originating from each illuminated nucleus
results in \textit{coherent forward scattering},
\index{Coherent forward scattering}%
in line with Huygens' principle.
\index{Huygens' principle}%

Another form of coherent scattering is \textit{Bragg scattering}.%
\index{Bragg scattering!by atomic lattices}%
However, Bragg scattering by atomic lattices
is large-angle, not small-angle scattering,
and therefore occurs outside the wavenumber range
covered in GISAS experiments.
Accordingly, it can be neglected in the analysis of GISAS data,
or at most, is taken into account as a loss channel.

Therefore,
we can neglect the atomic structure of $V(\r)$,
and perform some \textit{coarse graining} to
arrive at a \textit{continuum approximation}.
\index{Continuum approximation!neutron propagation}%
This is 
similar to the passage from
the microscopic to the macroscopic Maxwell equations.
The details are intricate \cite{Sea89,Lax51},
but the result \cite[eq.~2.8.32]{Sea89} looks very simple:
The macroscopic field equation
has still the form of a stationary Schrödinger equation,
\index{Schrodinger@Schrödinger equation!macroscopic}%
\begin{equation}\label{EmacrSchrodi}
  \left\{-\frac{\hbar^2}{2m}\Nabla^2+v(\r)-\hbar\omega\right\} \Psi(\r) = 0,
\end{equation}
where $\Psi$ now stands for the coherent superposition of
incident and forward scattered states,
and $v(\r)$ is the \textit{macroscopic optical potential}.
\index{Optical potential!macroscopic}%
This potential is weak, and slowly varying compared to atomic length scales.
It can be rewritten in a number of ways,
especially in terms of a
\textit{bound scattering length density}
\index{Scattering length density}%
$\rho_s(\r)$ \cite[eq.\ 2.8.37]{Sea89},
\begin{equation}
  v(\r)=\frac{2\pi \hbar^2}{m}\rho_s(\r),  
\end{equation}
or of a \textit{refractive index}~$n(\r)$
\index{Refractive index} % !vs scattering length density}%
\index{Index of refraction|see {Refractive index}}%
defined by
\begin{equation}\label{EnRefrIndx}
  n(\r)^2:=1-\frac{4\pi}{K^2}\rho_s(\r) = 1 -\frac{2m}{\hbar^2 K^2}v(\r).
\end{equation}
In the latter expression,
we introduced the \textit{vacuum wavenumber}~$K$,
which is connected with the frequency~$\omega$ through the
\textit{dispersion relation}
\begin{equation}
  \frac{\hbar^2 K^2}{2m} = \hbar\omega.
\end{equation}
Since we only consider stationary solutions~(\ref{Estationarywave}),
$\omega$ will not appear any further in our derivations.
Instead, we use~$K$ as the given parameter that characterizes the
incoming radiation.
In terms of $K$ and $n$,
the macroscopic Schrödinger equation (\ref{EmacrSchrodi})
can be rewritten as

\Important{\Box{
\begin{equation}\label{EnSchrodi}
  \left\{\Nabla^2+K^2n(\r)^2\right\}\Psi(\r) = 0.
\end{equation}
}}
This equation is the starting point for the analysis of all
small-angle scattering experiments,
whether under grazing incidence (GISAS) or not (regular SAS).

\index{Wave propagation!neutrons|)}%
\index{Neutrons!wave propagation|)}%

%%%%%%%%%%%%%%%%%%%%%%%%%%%%%%%%%%%%%%%%%%%%%%%%%%%%%%%%%%%%%%%%%%%%%%%%%%%%%%%%
\section{Neutron scattering in Born approximation}
%%%%%%%%%%%%%%%%%%%%%%%%%%%%%%%%%%%%%%%%%%%%%%%%%%%%%%%%%%%%%%%%%%%%%%%%%%%%%%%%

For slow neutrons, as for X-rays, the refractive index~$n$ is almost always
very close to~1.
This suggests a solution of the Schrödinger equation~(\ref{EnSchrodi})
by means of a perturbation expansion in powers of $n^2-1$.
This expansion is named after Max Born
who introduced it in quantum mechanics.\footnote
{It goes back to Lord Rayleigh
who devised it for the sound,
and later also applied it to electromagnetic waves.}

%===============================================================================
\subsection{The Born expansion}\label{SBorn}
%===============================================================================

\index{Born approximation|(}%

We rewrite the Schrödinger equation~(\ref{EnSchrodi})
once more so that it takes the form of a Helmholtz equation
\index{Helmholtz equation}%
with a perturbation term on the right side:
\begin{equation}\label{ESchrodiHelmholtz}
  \left(\Nabla^2+K^2\right)\Psi(\r)
  = \frac{\chi(\r)}{4\pi}\Psi(\r)
\end{equation}
with
\begin{equation}\label{EChiDef}
  \chi(\r) := \frac{4\pi}{K^2}\left(1-n^2(\r)\right).
\end{equation}
This definition just compensates (\ref{EnRefrIndx}) so that $\chi=\rho_s$.
Equation~(\ref{ESchrodiHelmholtz}) looks
like an inhomogeneous differential equation ---
provided we neglect for a moment that the unknown function~$\Psi$
reappears on the right side.
The homogeneous equation
\begin{equation}\label{EHelmholtzHomog}
  \left(\Nabla^2+K^2\right)\Psi_0(\r) = 0
\end{equation}
is solved by plane waves and superpositions thereof.
It applies in particular to the incident radiation
$\Psi_\text{i}$ that must be present in any scattering experiment.

For an isolated inhomogeneity,
\begin{equation}\label{EHelmholtzForGreen}
  \left(\Nabla^2+K^2\right)G(\r,\r') = \delta(\r-\r')
\end{equation}
\index{Green function!homogeneous material}%
is solved by the Green function\footnote
{Verification under the condition $\r\ne0$
is a straightforward exercise in vector analysis.
For the special case $\r=0$,
one encloses the origin in a small sphere
and integrates by means of the Gauss-Ostrogadsky divergence theorem.
This explains the appearance of the factor $4\pi$.}
\begin{equation}\label{EGreens1}
  G(\r,\r') = \frac{{\rm e}^{iK|\r-\r'|}}{4\pi |\r-\r'|},
\end{equation}
which is an outgoing spherical wave centered at $\r'$.
Convoluting this function with the given inhomogeneity $(\chi/4\pi)\Psi$,
we obtain the formal solution
of the full inhomogeneous equation~(\ref{ESchrodiHelmholtz})
\begin{equation}\label{EPsiFormal}
  \Psi(\r)
  = \Psi_\text{i}(\r)
  + \int\!{\rm d}^3r'\, G(\r,\r')
                     \frac{\chi(\r')}{4\pi}\Psi(\r').
\end{equation}
However, the integral kernel still contains the full solution~$\Psi$.
This implicit equation
can be resolved into an infinite series
by iteratively substituting the full right-hand side of~(\ref{EPsiFormal})
for the occurrence of $\Psi$ in the integrand.
This can be iterated to obtain an infinite series representation of $\Psi$.
Successive terms in this series contain rising powers of $\chi$.
Since $\chi$ is assumed to be small, the series is likely to converge.
In first Born approximation,
only the linear order in $\chi$ is retained,
\begin{equation}\label{EBorn}
  \Psi(\r)
  \doteq \Psi_\text{i}(\r)
  + \int\!{\rm d}^3r'\, G(\r,\r')\frac{\chi(\r')}{4\pi}
   \Psi_\text{i}(\r').
\end{equation}
This is practically always adequate for
material investigations with X-rays or neutrons,
where the aim is to 
deduce $\chi(\r')$ from the scattered intensity ${|\Psi(\r)|}^2$.
Since detectors are always placed at positions $\r$
that are not illuminated by the incident beam,
we are only interested in the scattered wave field
\begin{equation}\label{EBornS}
  \Psi_\text{s}(\r)
  :=
  \int\!{\rm d}^3r'\, G(\r,\r')\frac{\chi(\r')}{4\pi}\Psi_\text{i}(\r').
\end{equation}

\index{Born approximation|)}%

%===============================================================================
\subsection{Far-field approximation}
%===============================================================================

\index{Far-field approximation|(}%

We can further simplify (\ref{EBornS})
by making use of the fact that the detector location~$\r$
is far away from the sample volume $\Sample$.
Since the scattered wave $\Psi_\text{s}(\r)$
only depends on $\r$ through the Green function~$G(\r,\r')$,
we shall derive a far-field approximation for the latter.

We choose the coordinate origin within $\Sample$
so that the integral in~(\ref{EBornS}) runs over $\r'$ with $r'\ll r$.
This allows us to expand
\begin{equation}
  \left|\r-\r'\right|
  \doteq \sqrt{r^2-2\r\,\r'}
  \doteq r - \frac{\r\,\r'}{r}
  \equiv r - \frac{\k_\text{f} \r'}{K},
\end{equation}
where we have introduced the outgoing wave vector
\begin{equation}
  \k_\text{f}:=K\frac{\r}{r}.
\end{equation}
We apply this to~(\ref{EGreens1}),
\index{Green function!homogeneous material}%
and obtain in leading order the far-field Green function
\begin{equation}\label{EGreenFar}
  G_\text{far}(\r,\r')
  =\frac{{\rm e}^{iKr}}{4\pi r}\Psi^*_\text{f}(\r')
  \end{equation}
where
\begin{equation}
  \Psi_\text{f}(\r') := {\rm e}^{i\k_\text{f} \r'}
\end{equation}
is a plane wave propagating towards the detector.
With respect to $\r$, $G_\text{far}$ is an outgoing spherical wave.

The scattered wave~(\ref{EBornS})
becomes in far-field approximation 
\begin{equation}\label{EsandwichC}
  \Psi_\text{s,far}(\r)
  = \frac{{\rm e}^{iKr}}{r}
    \langle \Psi_\text{f}|\chi|\Psi_\text{i}\rangle,
\end{equation}
where we used Dirac notation for the transition matrix element
\index{Transition matrix}%
\begin{equation}\label{Etrama}
  \langle \Psi_\text{f}|\chi|\Psi_\text{i}\rangle
  := \int\!{\rm d}^3r\, \Psi^*_\text{f}(\r)\chi(\r)\Psi_\text{i}(\r).
\end{equation}
Making the standard assumption
that the incident radiation is a plane wave
\begin{equation}\label{EPsi0Plane}
  \Psi_\text{i}(\r)={\rm e}^{i \k_\text{i} \r}
\end{equation}
with $k_\text{i}=K$,
and introducing the \textit{scattering vector}
\index{Scattering vector}%
\begin{equation}
  \v{q}:=\k_\text{i}-\k_\text{f},
\end{equation}
we can rewrite (\ref{EsandwichC}) as
\begin{equation}\label{EBornQ}
  \langle \Psi_\text{f}|\chi|\Psi_\text{i}\rangle
  = \int\!{\rm d}^3r\, {\rm e}^{i\v{q}\,\r}\chi(\r)
  =: \chi(\v{q}),
\end{equation}
which shows that neutron scattering,
in first-order Born approximation,
measures the Fourier transform
of the optical potential.
\index{Optical potential!Fourier transform}%

\index{Far-field approximation|)}%

%===============================================================================
\subsection{Differential cross section}
%===============================================================================

Above, we said somewhat sloppily
that a scattering experiment measures intensities~${|\Psi(\r)|}^2$.
We shall now make this more rigorous.
In the case of neutron scattering,
one actually measures the probability flux.
We define it in arbitrary relative units as
\begin{equation}
  \v{J}(\r) := \Psi^*\frac{\Nabla}{2i}\Psi - \Psi\frac{\Nabla}{2i}\Psi^*.
\end{equation}
\index{Flux!incident and scattered}%
With (\ref{EPsi0Plane}), the incident flux is
\begin{equation}
  \v{J}_\text{i} = \k_\text{i}.
\end{equation}
With (\ref{EsandwichC}), the scattered flux at the detector is
\begin{equation}\label{EJr}
  \v{J}(\r)
  = \v{\hat r}\frac{K}{r^2}
    {\left|\langle\Psi_\text{f}|\chi|\Psi_\text{i}\rangle\right|}^2.
\end{equation}
The ratio of the scattered current hitting an infinitesimal detector area
$r^2{\rm d}\Omega$ to the incident flux is expressed as a
\textit{differential cross section}
\index{Cross section}%
\begin{equation}
  \frac{{\rm d}\sigma}{{\rm d}\Omega}
  := \frac{r^2 J(\r)}{J_\text{i}}.
\end{equation}
The resulting equation

\Important{\Box{
\begin{equation}\label{Exsection}
  \frac{{\rm d}\sigma}{{\rm d}\Omega}
  =  {\left|\langle\Psi_\text{f}|\chi|\Psi_\text{i}\rangle\right|}^2.
\end{equation}
}}
yields quite generically the differential cross section of elastic scattering
in first order Born approximation.
As we shall see below,
it holds not only for plane waves governed
by the vacuum Helmholtz equation~(\ref{EHelmholtzHomog}),
but also for distorted waves.

In the plane-wave case, we can insert (\ref{EBornQ}) to obtain
\begin{equation}\label{Ecross1}
  \frac{{\rm d}\sigma}{{\rm d}\Omega}
  = {\left| \chi(\v{q}) \right|}^2.
\end{equation}
Thus the differential cross section is just the squared modulus
of the Fourier transform 
\index{Scattering length density}%
of the scattering-length density.


%%%%%%%%%%%%%%%%%%%%%%%%%%%%%%%%%%%%%%%%%%%%%%%%%%%%%%%%%%%%%%%%%%%%%%%%%%%%%%%%
\section{Scattering under grazing incidence}\label{Sdwba}
%%%%%%%%%%%%%%%%%%%%%%%%%%%%%%%%%%%%%%%%%%%%%%%%%%%%%%%%%%%%%%%%%%%%%%%%%%%%%%%%

%===============================================================================
\subsection{Wave propagation in $\v{2+1}$ dimensions}
%===============================================================================

Reflectometry and grazing-incidence scattering
are designed for the investigation of surfaces, interfaces, and thin layers,
or most generically:
samples with a $2+1$ dimensional structure,
that is translationally invariant in $x$ and $y$ direction,
while being nontrivially modulated in $z$ direction.
By convention,
the $z$ axis points upwards, hence out of the sample towards the
vacuum (or air) half space where the incident radiation comes from.
% TODO: add figure

Vertical modulations of the refractive index $n(\r)$
cause refraction and reflection of an incident plane wave.
\index{Glancing angle}%
\index{Refraction}%
\index{Reflection}%
For small glancing angles,
these distortions can be arbitrary large,
up to the limiting case of total reflection,
even though $\delta n$ is only of the order $10^{-5}$ or smaller.
Such zeroth-order effects cannot be accounted for
by perturbative scattering theory.
Instead, we need to deal with refraction and reflection
at the level of the wave propagation equation.
We move the vertical variations of the squared refractive index
to the left-hand side of the Schrödinger equation~(\ref{EnSchrodi},
\begin{equation}\label{EHelmholtzGraded}
  \left\{\Nabla^2+K^2\nz(z)\right\}\Psi(\r)
  = \frac{\chi(\r)}{4\pi}\Psi(\r),
\end{equation}
where the overline indicates an horizontal average.
Accordingly, the perturbation
\begin{equation}\label{EChiGraded}
  \chi(\r) := \frac{4\pi}{K^2}\left(\nz(z)-n^2(\r)\right),
\end{equation}
only accounts for horizontal fluctuations of the refractive index.
Wave propagation,
unperturbed by~$\chi$, but including refraction and reflection effects,
obeys the homogeneous equation
\begin{equation}\label{EHelmholtzGradedHomog}
  \left\{\Nabla^2+K^2\nz(z)\right\}\Psi_\text{i}(\r) = 0,
\end{equation}
which suggests the factorization ansatz
\begin{equation}\label{Ekpar}
\Psi_\text{i}(\r) = {\rm e}^{i \k_\parallel\r_\parallel} \phi(z).
\end{equation}
The in-plane wave vector $\k_\parallel$ remains constant
as initialized by the incoming beam.
The vertical wave function must fulfill
\begin{equation}\label{Ewavez}
\left\{\partial_z^2 + K^2\nz(z) - k_\parallel^2 \right\} \phi(z) = 0.
\end{equation}
When an incident plane wave,
travelling downwards with
$\phi(z)={\rm e}^{-ik_\perp z}$,
impinges on a sample with $\nz(z)\ne 1$,
then the wave is partly reflected ($-k_\perp$ reversed into $+k_\perp$)
and partly refracted
($k_\perp$ changing while $\k_\parallel$ stays constant,
resulting in a change of glancing angle).
Similarly, reflection and refraction occur
whenever $\nz(z)$ varies within the sample.
As a result, at any $z$ within the zone where $\nz(z)$ varies,
the vertical wave function $\phi(z)$ is composed of a
downward travelling component $\phi^-(z)$
and an upward travelling component $\phi^+(z)$.

For a graded refractive index
\index{Refractive index!graded}%
$\nz$ that is a smooth function of~$z$,
the differential equation~(\ref{Ewavez}) is best solved
with the WKB method.\footnote
{Also called \textit{semiclassical approximation} or
\textit{phase integral method},
originally developed
by Liouville (1837), Green (1837), Lord Rayleigh (1912), and Jeffreys (1923),
and named WKB method after three independent
quantum-mechanical papers by
Wentzel (1926), Kramers (1926), Brillouin (1926).
See any textbook on quantum mechanics for more information.}
\index{WKB method}%
\index{Semiclassical approximation|see {WKB method}}%
\index{Phase integral method|see {WKB method}}%
If otherwise $\nz(z)$ is discontinuous at some interface $z=z_j$,
then the limiting values of $\phi^-(z)$ and $\phi^+(z)$
on approaching $z_j$ from above or below
are connected to each other through Fresnel's
transmission and reflection coefficients.
\index{Fresnel coefficients}%

%===============================================================================
\subsection{Reciprocity and the distorted-wave Born approximation}
%===============================================================================

\index{Distorted-wave Born approximation|(}%
\index{DWBA|see {Distorted-wave Born approximation}}%

The standard form of the Born approximation,
as presented in Sect.~\ref{SBorn},
combines an approximation scheme
(computing (\ref{EPsiFormal}) by iteration)
with an assumption (the incident field is a plane wave)
and an analytic result
(in far-field approximation,
the Green function of the Helmholtz equation is a plane wave
with respect to the locus of scattering).
These three elements must not necessarily go together.
We can apply the very same approximation scheme,
even if the incident field is not a plane wave,
but a \textit{distorted wave},
namely a superposition of downwards and upwards travelling plane waves,
as derived in the previous section.
This is the core idea
of the \textit{distorted-wave Born approximation} (DWBA).\footnote
{The distorted-wave Born approximation
was originally devised by Massey and Mott (ca 1933)
for collisions of charged particles.}
% Schiff (^3 1968, p 327) cites
% Mott, Massey, The Theory of Atomic Collisions, p 100, Oxford 1933;
% There are also several papers by Massey and Mott from about 1933.

There is one difficulty, though:
how to substitute the Green function (\ref{EGreens1})?
\index{Green function!vertically structured material}%
It was derived in Sect.~\ref{SBorn} quite specifically
for a homogeneous material.
Computing it in closed form for a more generic wave equation
like~(\ref{EHelmholtzGradedHomog}) is far more difficult,
if not outright impossible.
Fortunately, we do not need the full Green function,
and computing it would be wasted effort,
because in the end
we only need its asymptotic far-field value at a detector position $\rD$.
Instead of propagating a field from a scattering center $\rS$
into all directions, and finally evaluating a plane wave asymptote at~$\rD$,
we better propagate a plane wave backwards from $\rD$
into our structured material,
and finally evaluate the field at~$\rS$.
The equivalence of these two computations is ensured
by the \textit{reciprocity}
\index{Reciprocity|(}%
\index{Green function!reciprocity}%
of Green functions for self-adjoint differential equations.

Let us derive the reciprocity theorem
for a generic stationary Schrödinger equation
with an isolated inhomogeneity,
\begin{equation}\label{EgsSchrodi}
  \left\{\Nabla^2+v(\r)\right\}G(\r,\rS) = \delta(\r-\rS).
\end{equation}
We assume that $\rS$ lies within a structured material~$\Sample$.
Outside $\Sample$,
the potential~$v(\r)$ has the constant value~$K^2$
so that (\ref{EgsSchrodi})
reduces to the Helmholtz equation
\begin{equation}\label{EgsSchrodiHelmh}
  \left\{\Nabla^2+K^2\right\}G(\r,\rS) = 0
  \text{~~for~~}\r\notin\Sample.
\end{equation}
We introduce the adjoint Green function~$B$
that originates from a source term at the detector location
and obeys
\begin{equation}\label{EgsSchrodiAdj}
  \left\{\Nabla^2+v(\r)\right\}B(\r,\rD) = \delta(\r-\rD).
\end{equation}
We also introduce the auxiliary vector field
\begin{equation}
  \v{X}(\r,\rS,\rD):=B(\r,\rD)\Nabla G(\r,\rS) - G(\r,\rS)\Nabla B(\r,\rD).
\end{equation}
We inscribe $\Sample$, $\rD$, and the origin of the coordinate system
into a sphere $\Sphere$ with radius~$R$,
and compute the volume integral
\begin{equation}\label{Eprerecipro}
  \begin{array}{lcl}
    I(\rS,\rD)
  &=& \displaystyle\int_\Sphere\!{\rm d}^3r\,\Nabla \v{X}(\r,\rS,\rD)
  \\[1.8em]
  &=& \displaystyle\int_\Sphere\!{\rm d}^3r\,\left(
    B\Nabla^2 G- G\Nabla^2 B \right)
  \\[1.4em]
  &=&  B(\rS,\rD) - G(\rD,\rS).
  \end{array}
\end{equation}
Alternatively, we can compute $I$ as a surface integral
\begin{equation}
  I(\rS,\rD)
  =\displaystyle\int_{\partial\Sphere}{\rm d}\v{\sigma}\,\v{X}(\r,\rS,\rD)
  =\displaystyle\int_{\partial\Sphere}{\rm d}{\sigma}\,
       \left(B\partial_R G - G\partial_R B\right).
\end{equation}
On the surface $\partial\Sphere$, which lies outside $\Sample$,
$B$ and $G$ are outgoing wave fields that obey the Helmholtz equation.
Solutions of this equation in spherical coordinates
have a well-known series expansion.
We send $R\to\infty$ so that we need only to retain the lowest order,
\begin{equation}
   G(\r(R,\vartheta,\varphi),\rS)
   \doteq \frac{{\rm e}^{iKR}}{4\pi R} g(\vartheta,\varphi),
\end{equation}
and similarly 
\begin{equation}
   G(\r(R,\vartheta,\varphi),\rD)
   \doteq \frac{{\rm e}^{iKR}}{4\pi R} b(\vartheta,\varphi).
\end{equation}
The functions $g$ and $b$ can be further expanded into spherical harmonics,
but this is of no interest here.
The decisive point is the factorization of $G$ and $B$
and their common $R$ dependence.
It follows at once that
\begin{equation}
  I(\rS,\rD)
  =\displaystyle\int_{\partial\Sphere}{\rm d}\sigma\,
       (\text{$R$-dependent})(bg-gb)
  = 0.
\end{equation}
From (\ref{Eprerecipro}) we obtain the \textit{reciprocity theorem}
\begin{equation}\label{Ereci}
  G(\rD,\rS) = B(\rS,\rD).
\end{equation}
To obtain the forward-propagating Green function~$G$,
it is sufficient to determine the adjoint Green function~$B$
that traces a neutron back from the detector position~$\rD$
to a scattering center~$\rS$.
\index{Reciprocity|)}%

The solution $B(\r,\rD)$ of (\ref{EgsSchrodiAdj})
is analogous to (\ref{EGreens1}).
Choosing the origin inside~$\Sample$, 
and assuming that $\rD$ lies far outside~$\Sample$,
we can copy the far-field expansion from~(\ref{EGreenFar})
to obtain
\index{Far-field approximation}%
\begin{equation}\label{EBFar}
  B_\text{far}(\r,\rD)
  =\frac{{\rm e}^{iKr_\text{D}}}{4\pi r_\text{D}}{\rm e}^{-i\k_\text{f} \r}.
\end{equation}
When this backward propagating plane waves impinges on the sample,
it undergoes reflection and refraction in exactly the same way as
the incident plane wave ${\rm e}^{i\k_\text{i}\r}$.
Therefore,
 (\ref{EBFar}) admits a generalization that also holds inside $\Sample$:
\begin{equation}\label{EBFull}
  B_\text{far}(\r,\rD)
  = \frac{{\rm e}^{iKr_\text{D}}}{4\pi r_\text{D}}\Psi^*_\text{f}(\r).
\end{equation}
Applying now the reciprocity theorem (\ref{Ereci}),
we can conclude that the far-field Green function
is still given by (\ref{EGreenFar}),
though $\Psi_\text{f}$ no longer is a plane wave.
Accordingly,
the scattered far-field is still given by (\ref{EsandwichC}),
and the differential cross section by (\ref{Exsection}).

Since both the incident
and the scattered distorted wave function
are composed of downward and upward propagating waves,
\begin{equation}
  \Psi_w(\r)
  = \Psi^-_w(\r) + \Psi^+_w(\r)\text{~for~}w\in\{\text{i},\text{f}\},
\end{equation}
the matrix element (\ref{Etrama})
can be expanded into four terms,

\Important{\Box{
\begin{equation}\label{EtmDWBA}
  \langle \Psi_\text{f}|\chi|\Psi_\text{i}\rangle
  = \langle \Psi^-_\text{f}|\chi|\Psi^-_\text{i}\rangle
  + \langle \Psi^-_\text{f}|\chi|\Psi^+_\text{i}\rangle
  + \langle \Psi^+_\text{f}|\chi|\Psi^-_\text{i}\rangle
  + \langle \Psi^+_\text{f}|\chi|\Psi^+_\text{i}\rangle.
\end{equation}
}}
This equation contains the essence of
the distorted-wave Born approximation
for small-angle scattering under grazing incidence,
and is the base for all scattering models implemented in BornAgain.
In the following, we will use the shorthand notation
\begin{equation}\label{EtmDWBAsum}
  \langle \Psi_\text{f}|\chi|\Psi_\text{i}\rangle
  = \sum_{\pm_\text{f}} \sum_{\pm_\text{i}}
    \langle \Psi^\pm_\text{f}|\chi|\Psi^\pm_\text{i}\rangle.
\end{equation}
Note that $\langle \Psi_\text{f}|\chi|\Psi_\text{i}\rangle$
appears as a squared modulus
in the differential cross section~(\ref{Exsection}).
Therefore, the four terms of (\ref{EtmDWBA}) can interfere with each other,
which adds to the complexity of GISAS patterns.

\index{Distorted-wave Born approximation|)}%

%%%%%%%%%%%%%%%%%%%%%%%%%%%%%%%%%%%%%%%%%%%%%%%%%%%%%%%%%%%%%%%%%%%%%%%%%%%%%%%%
\section{Multilayer systems}
%%%%%%%%%%%%%%%%%%%%%%%%%%%%%%%%%%%%%%%%%%%%%%%%%%%%%%%%%%%%%%%%%%%%%%%%%%%%%%%%

\index{Multilayer systems}%

%===============================================================================
\subsection{DWBA}
%===============================================================================

In many applications of GISAS,
the investigated sample consists of a number of distinct layers.
For the time being,
no other systems are supported in BornAgain.\footnote
{Please contact us if need arises for modelling systems with
smoothly varying~$\nz(z)$.
\index{Refractive index!graded}%
}
Within one layer~$j$, the average refractive index is constant,
$\nz(z)=\nzj$ for $z\in\mathcal{Z}_j$, where
$\mathcal{Z}_j$ denotes an interval on the vertical coordinate axis.
Within a layer,
the downward and upward propagating waves are plane,
\begin{equation}
  \Psi^\pm_{wj}(\r)=A^\pm_{wj}{\rm e}^{i\k^\pm_{wj}\r}.
\end{equation}
\index{Fresnel coefficients}%
\index{Transmission|see {Fresnel coefficients}}%
\index{Reflection|seealso {Fresnel coefficients}}%
Here and in the following,
the index~$w$ can take the values i and~f.
The amplitudes $A$ are often written with distinct letters
T and~R to designate the transmitted or reflected beam,
\begin{equation}
  T_{wj} := A^-_\text{wj},\quad
  R_{wj} := A^+_\text{wj}.
\end{equation}
The wave vector can be decomposed as
\begin{equation}
  \k^\pm_{wj}= \k_{\parallel w} \pm k_{\perp wj}\v{\hat z}.
\end{equation}
As explained in connection with~(\ref{Ekpar}),
the in-plane wave vector $\k_{\parallel w}$ remains constant
across layer interfaces.
The vertical wavenumber is obtained from (\ref{Ewavez}),
\begin{equation}
  k_{\perp wj} = \sqrt{K^2 \nzj - k_{\parallel w}^2}.
\end{equation}
Restricting (\ref{EBornQ}) to one layer,
the Fourier transform of the refractive index fluctuations
shall be written as
\begin{equation}\label{Echij}
  \chi_j(\v{q})
  := \int_{z\in\mathcal{Z}_j}\!{\rm d}^3r\, {\rm e}^{i\v{q}\,\r}\chi(\r).
\end{equation}
We can then write the DWBA transition matrix~(\ref{EtmDWBAsum}) as
a sum over all layers,
\index{Distorted-wave Born approximation!multilayer}%
\Important{\Box{
\begin{equation}\label{EtmDWBAft}
  \langle \Psi_\text{f}|\chi|\Psi_\text{i}\rangle
  = \sum_{j} \sum_{\pm_\text{f}} \sum_{\pm_\text{i}}
    A^\pm_{\text{i}j} A^{\pm *}_{\text{f}j}
     \chi_j(\k^\pm_{\text{i}j} - \k^\pm_{\text{f}j}).
\end{equation}
}}

%===============================================================================
\subsection{Transmission and reflection coefficients}
%===============================================================================

The coefficients $A^\pm_{wj}$
can be calculated using the Parratt formalism \cite{Par54}
or the matrix method \cite{BoWo99}.

..... And how are they actually computed in BornAgain? .....

%===============================================================================
\subsection{Embedded particles}
%===============================================================================

\index{Particles!embedded|(}%
\index{Nanoparticles|see Particles}%
\index{Mesoparticles|see Particles}%
In many important GISAS applications,
fluctuations of the refractive index are due to particles
of mesoscopic size (nanometer to micrometer).
These particles are either embedded into a material layer,
or attached to the surface of the sample.
In BornAgain, the latter case must be construed as a special form of the former
by defining an extra layer on top of the regular surface that
consists of vacuum (or air) with embedded particles.

For the time being,
we consider scattering from only one layer;
therefore we can omit the layer index~$j$ from the following considerations.
The layer has a thickness~$d$ and contains $N$ particles per area~$A$.
% hence the particles have a number density of $\nu=N/(Ad)$.
Particle~$i$, located at $\r_i$,
has a volume~$V_i$ and scattering length density~$\rho_i(\r-\r_i)$.
Outside the volumes occupied by particles,
the scattering length has a constant density of~$\chi_0$.
The average scattering length density of the entire layer is then
\begin{equation}\label{Emulaynav}
  \begin{split}
  \overline{\rho}
      &= \frac{\sum_i \int_{V_i}\!{\rm d}^3r\; \rho_i(\r-\r_i)
               + (Ad-\sum_i V_i)\rho_0}
             {Ad}\\[1em]
      &= \rho_0 + \frac{\sum_i \int_{V_i}\!{\rm d}^3r\, [\rho_i(\r-\r_i)-\rho_0]}
             {Ad}
  \end{split}
\end{equation}
To predict the GISAS pattern,
we need to compute the Fourier transform (\ref{Echij}) of
$\chi(\r)=\rho(\r)-\overline{\rho}$.
There are two contributions to the Fourier integrand:
from the embedded particles, and from the embedding matrix,
which has the structure of an Emmental cheese:
homogeneous except for holes where the particles are.
Fortunately, we do not need to compute the Fourier transform
of the cheese:
Analogous to (\ref{Emulaynav}),
we can rearrange the Fourier integral as
\begin{equation}\label{Echijre}
  \chi_j(\v{q})
  = \int_{z\in\mathcal{Z}_j}\!{\rm d}^3r\, {\rm e}^{i\v{q}\,\r}
    \left\{ \left(\rho_0-\overline{\rho}\right)
           + \sum_i \left[(\rho_i(\r-\r_i)-\rho_0\right] \right\}
  =: \chi^0_j + \chi^\text{p}_j.
\end{equation}
The first term is the shape transform of the entire layer
\begin{equation}
  \chi^0_j(\v{q})
   = \int_{z\in\mathcal{Z}_j}\!{\rm d}^3r\, {\rm e}^{i\v{q}\,\r}
      \left(\rho_0-\overline{\rho}\right)
   = \delta(\v{q}_\parallel)\;d\;\text{sinc}(q_\parallel d/2).
\end{equation}
For the second term,
we abbreviate
\begin{equation}
  \delta\rho_i(\r):=\rho_i(\r)-\rho_0,
\end{equation}
and we introduce the form factor of a single particle
\begin{equation}
  F_i(\v{q}) = \int\!{\rm d}^3r\, {\rm e}^{i\v{q}\,\r} \delta\rho_i(\r)
\end{equation}
We can then rearrange to obtain
\begin{equation}
  \chi^\text{p}_j(\v{q})
  = \sum_i {\rm e}^{i\v{q}\,\r_i} F_i(\v{q}).
\end{equation}


%===============================================================================
\subsection{OLD STUFF}
%===============================================================================

... the same techniques as demonstrated in section \ref{sec:ba} can be applied, leading to the following expression for the expectation value of the scattering cross-section:
\begin{align*}
  & \left\langle \frac{d\sigma}{d\Omega}(\k_\text{i},\k_\text{f}) \right\rangle_{\text{Off-specular}}  \\
  & = \sum_\alpha p_\alpha \left\lvert \curlf_\alpha(\k_{j,i},\k_{j,f}, R_{\alpha,z})\right\rvert ^2 + \frac{\rho_S}{S}\sum_{\alpha,\beta} p_\alpha p_\beta \curlf_\alpha (\k_{j,i},\k_{j,f}, R_{\alpha,z})\curlf_\beta^*(\k_{j,i},\k_{j,f}, R_{\beta,z}) \\
  & \times \iint_S d^2\r_\alpha^\parallel d^2\r_\beta^\parallel \ppcf{\alpha}{\beta}{R^\parallel} \exp \left[ i\v{q}_{j\parallel}\cdot (\r_\alpha^\parallel - \r_\beta^\parallel ) \right].
\end{align*}

The main differences with respect to the cross-section in the Born approximation are:
\begin{enumerate}
  \item The particle form factor now consists of a more complex expression and now depends on both incoming and outgoing wave vectors and also on the $z$-coordinate of the particle;
  \item Since the $z$-coordinate of the particles is implicitly included in its formfactor, the position integrals only run over $x$- and $y$-coordinates and the volume and density gets replaced with the surface area and surface density respectively.
\end{enumerate}

....
Consider a scattering volume $V$, containing $N$ scattering centers with shape functions $S^i(\r)$, positions $\r^i$ and scattering length density $\rho_s$ (relative to the ambient material).

The differential cross-section (per scattering center) is then given by:
\begin{equation}\label{Exsec1}
  \frac{d\sigma}{d\Omega}(\v{q}) = \frac{1}{N}\left\lvert \int_V \rho_s(\r) e^{i\v{q}\cdot\r} d^3\r \right\rvert ^2.
\end{equation}

