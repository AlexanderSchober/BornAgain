%%%%%%%%%%%%%%%%%%%%%%%%%%%%%%%%%%%%%%%%%%%%%%%%%%%%%%%%%%%%%%%%%%%%%%%%%%%%%%%%
%%
%%   BornAgain User Manual
%%
%%   homepage:   http://www.bornagainproject.org
%%
%%   copyright:  Forschungszentrum Jülich GmbH 2015
%%
%%   license:    Creative Commons CC-BY-SA
%%   
%%   authors:    Scientific Computing Group at MLZ Garching
%%               C. Durniak, M. Ganeva, G. Pospelov, W. Van Herck, J. Wuttke
%%
%%%%%%%%%%%%%%%%%%%%%%%%%%%%%%%%%%%%%%%%%%%%%%%%%%%%%%%%%%%%%%%%%%%%%%%%%%%%%%%%


\chapter{Scattering Theory}  \SecLabel{ScatTheory}

%===============================================================================
\subsection{Distorted Wave Born Approximation} \SecLabel{sect:dwba}
%===============================================================================
\index{Distorted wave Born approximation}

The Born approximation fails when multiple reflections and refractions have to be taken into account at interfaces because of the presence of underlying layers of materials and the closeness of  the incident angle $\alpha_i$ to the critical angle of total external reflection $\alpha_c$. The first order correction to the scattering theory is the Distorted Wave Born Approximation (DWBA), whereas the Born approximation is the zeroth order. \\
The collective effects between the particles are not considered in this section. They have been described in~\SecRef{sect:interf}.  We also do not take any polarization effects into account. \\

 In the DWBA, the form factor of a particle in a multilayer system is given by

\begin{align}
F_{\rm{DWBA}} (\vect{k}_i,\vect{k}_f, r_z) & = T_i T_f F_{\rm{BA}} (\vect{k}_i-\vect{k}_f) e^{i (k_{i,z}-k_{f,z}) r_z} + R_i T_f F_{\rm{BA}}(\vect{\widetilde{k}}_i-\vect{k}_f) e^{i(-k_{i,z}-k_{f,z})r_z}
 \nonumber \\
  &+ T_i R_f F_{\rm{BA}}(\vect{k}_i-\vect{\widetilde{k}}_f)e^{i(k_{i,z}+k_{f,z})r_z} + R_iR_fF_{\rm{BA}} (\vect{\widetilde{k}}_i-\vect{\widetilde{k}}_f)e^{i(-k_{i,z}+k_{f,z})r_z} \; , \label{eq:dwbageneral}
\end{align}
where $F_{\rm{BA}}$ is the expression of the form factor in the Born approximation, $r_z$ is the $z$-coordinate of the particle's position (measured from the bottom of the particle), $\vect{k}_i=(k_{i,x}, k_{i,y}, k_{i,z})$ $\vect{k}_f=(k_{f,x}, k_{f,y}, k_{f,z})$ are the incident and scattered wave vectors in air, respectively \cite{RaSS95}. With a tilde (\~{}), these wavevectors components are evaluated in the multilayer system (the refractive indices of the different constituting materials have to be taken into account). 
$T_i$, $T_f$, $R_i$, $R_f$ are the transmission and reflection coefficients for the incident wave (index $i$) or the scattered one (index $f$). These coefficients can be calculated using the Parratt formalism \cite{Par54} or the matrix method \cite{BoWo99}. $\vect{k}_i-\vect{k}_f$ is equal to the scattering vector $\vect{q}$ and the $z$-axis is pointing upwards.\\

%===============================================================================
\subsection{Green operators and the $T$-matrix}
%===============================================================================

For a particle, governed by the Schr\"odinger equation with Hamiltonian $H = H_0 + V$, the time-independent scattering theory formally consists of solving the eigenvalue equations:

\begin{equation*}
  H\Psi_\alpha = E_\alpha\Psi_\alpha \; ,
\end{equation*}
with $E$ the scalar energy eigenvalue of the eigenstate $\Psi(E)$.\\
If the solutions of the free (or unperturbed) Hamiltonian $H_0$ are known:
\begin{equation*}
  H_0\Psi_{0\alpha} = E_\alpha\Psi_{0\alpha} \; ,
\end{equation*}
one can write the solutions of the full Hamiltonian in terms of these asymptotic states and Green operators:

\begin{align*}
  \Psi^\pm_\alpha &= \Psi_{0\alpha} + G^\pm_0 V \Psi^\pm_\alpha  \\
  & = \Psi_{0\alpha} + G^\pm V  \Psi_{0\alpha} \; ,
\end{align*}

where the Green operators are defined as:
\begin{align*}
  G^\pm_0(E) &= (E-H_0\pm i\epsilon )^{-1}  \\
  G^\pm (E) &= (E-H\pm i\epsilon )^{-1} \; .
\end{align*}
In these equations, the upper index or sign refers to the state corresponding with the free state $\Psi_{0\alpha}$ at time $t\rightarrow - \infty$ (and vice-versa for the lower sign). Since the solutions of the eigenvalue equations, both for the unperturbed as for the full Hamiltonian, are dependent on the energy eigenvalue $E$, the index $\alpha$ is assumed to include this value (and possibly other quantum numbers).

The transition amplitude between two asymptotic states is given by the $S$-matrix elements, defined as:
\begin{align}
  S_{\alpha\beta} &\equiv \braket{\Psi_{0\beta}}{S}{\Psi_{0\alpha}} \nonumber\\
  & \equiv \braketnoop{\Psi_\beta^-}{\Psi_\alpha^+} \; .
  \label{<++>}
\end{align}

The $S$-matrix can be decomposed into a delta function, representing the absence of scattering, and a $T$-matrix that encodes the scattering part, caused by the potential $V$:
\begin{equation*}
  S_{\alpha\beta} = \delta (E_\alpha - E_\beta)\delta_{\alpha\beta} - 2\pi i \delta (E_\alpha - E_\beta) T^\pm_{\alpha\beta} \; ,
\end{equation*}
with
\begin{align*}
  T^+_{\alpha\beta} & = \braket{\Psi_{0\beta}}{V}{\Psi^+_\alpha} \\
  T^-_{\alpha\beta} & = \braket{\Psi^-_\beta}{V}{\Psi_{0\alpha}} \; .
\end{align*}
On the energy shell $E_\alpha = E_\beta$, one has $T^+_{\alpha\beta}= T^-_{\alpha\beta}$, so that both formulations are equivalent.

By expanding the eigenstates $\Psi^\pm_\alpha$ in these equations, the $T$-matrix elements (on-shell) can be expressed as:
\begin{equation*}
  T^\pm_{\alpha\beta} = V + VG^+ V \; .
\end{equation*}


%===============================================================================
\subsection{Momentum representation and the scattering cross-section}
%===============================================================================

The previous general formulas can also be presented in a momentum (and position) eigenbasis, defined by:
\begin{align*}
  \oper{P}\ket{\vect{k}} & = \hbar \vect{k} \ket{\vect{k}} \\
  \braketnoop{\vect{k'}}{\vect{k}} & = \delta(\vect{k'}-\vect{k}) \\
  1 &= \int d^3\vect{k} \ket{\vect{k}} \bra{\vect{k}} \\
  1 &= \int d^3\vect{r} \ket{\vect{r}} \bra{\vect{r}} \\
  \braketnoop{\vect{r}}{\vect{k}} &= (2\pi)^{-3/2} \exp (i\vect{k}\cdot\vect{r}) \; ,
\end{align*}
where the normalization in the last equation follows from the other definitions.

The wavefunction that evolves from a momentum eigenstate $\ket{\vect{k}_i}$ can then be written as:
\begin{equation*}
  \braketnoop{\vect{r}}{\Psi^+} = \braketnoop{\vect{r}}{\vect{k}_i} + \braket{\vect{r}}{G_0^+T}{\vect{k}_i} \; ,
\end{equation*}
which in the far-field limit becomes:
\begin{align*}
  \braketnoop{\vect{r}}{\Psi^+} &= (2\pi)^{-3/2}\left[ e^{i\vect{k}_i\cdot\vect{r}} - \frac{4\pi^2m}{\hbar^2}\cdot\frac{e^{ik_f r}}{r} \braket{\vect{k}_f}{T}{\vect{k}_i} \right]  \\
  &\equiv (2\pi)^{-3/2}\left[ e^{i\vect{k}_i\cdot\vect{r}} + f(\theta,\phi)\frac{e^{ik_f r}}{r} \right] \; ,
\end{align*}
where $r,\theta,\phi$ are the polar coordinates of $\vect{r}$,
and the scattering amplitude was defined as
\begin{equation*}
  f(\theta, \phi) = -\frac{4\pi^2m}{\hbar^2}\braket{\vect{k}_f}{T}{\vect{k}_i} \; .
\end{equation*}

The amount of particles per unit time that are scattered in a small solid angle $d\Omega$ in direction $\vect{k}_f$ will then be (still in the far-field limit):
\begin{equation*}
  dI_{scat} = J_0\lvert f(\theta,\phi)\rvert^2d\Omega \; ,
\end{equation*}
where $J_0$ denotes the incident flux density. The scattering cross-section is defined as:
\begin{equation*}
  \frac{d\sigma}{d\Omega}\equiv \frac{dI_{scat}}{J_0 d\Omega} = \lvert f(\theta,\phi) \rvert^2 \; .
\end{equation*}


%%%%%%%%%%%%%%%%%%%%%%%%%%%%%%%%%%%%%%%%%%%%%%%%%%%%%%%%%%%%%%%%%%%%%%%%%%%%%%%%
\section{Small angle approximation}
%%%%%%%%%%%%%%%%%%%%%%%%%%%%%%%%%%%%%%%%%%%%%%%%%%%%%%%%%%%%%%%%%%%%%%%%%%%%%%%%

In the case of pure nuclear scattering, the Hamiltonian describing a neutron in a scattering experiment, is given by $H = -\dfrac{\hbar^2}{2m}\Delta + V$, where

\begin{equation*}
  V = \frac{2\pi \hbar^2}{m}\rho_s(\vect{r}) \; ,
\end{equation*}
with $\rho_s(\vect{r})$ the scattering length density of the sample. This scattering length density typically consists of a sum of weighted delta-functions, peaked at the atomic positions of the sample. For small scattering angles, the Bragg condition will not be fulfilled and the scattering length density may be replaced by a continuous function, representing the average scattering length density. In this case, one can define a refractive index, which in general will also be a continuous function of the position in the sample:
\begin{equation*}
  n^2(\vect{r}) \equiv 1 - \frac{4\pi}{k_0^2}\rho_s(\vect{r}) \; ,
\end{equation*}
with $k_0$ the wavevector in vacuum, or alternatively $k_0 = 2\pi /\lambda$, with $\lambda$ the de Broglie wavelength of the neutron.


Substituting this refractive index in the potential then gives:
\begin{equation*}
  V(\vect{r}) = \frac{\hbar^2}{2m}k_0^2(1-n^2(\vect{r})) \; .
\end{equation*}

Using these definitions, one can rescale the Hamiltonian with a factor $2m/\hbar^2$, such that
\begin{align*}
  \widetilde{H} &\equiv -\Delta + \widetilde{V} \nonumber \\
  \widetilde{V}(\vect{r}) &\equiv 4\pi\rho_s(\vect{r}) = k_0^2(1-n^2(\vect{r})) \; .
\end{align*}
It should be noted that this Hamiltonian implicitly contains the energy eigenvalue ($E_{k_0}=(\hbar k_0)^2/2m$), so that it can only be used in the time-independent Schr\"odinger equation $H\Psi_\alpha = E_\alpha \Psi_\alpha$.

The $T$-matrix then also becomes rescaled and the scattering amplitude becomes:
\begin{equation*}
  f(\theta, \phi) = -2\pi^2 \braket{\vect{k}_f}{\widetilde{T}}{\vect{k}_i} \; .
\end{equation*}



%%%%%%%%%%%%%%%%%%%%%%%%%%%%%%%%%%%%%%%%%%%%%%%%%%%%%%%%%%%%%%%%%%%%%%%%%%%%%%%%
\section{Born approximation} \label{sec:ba}
%%%%%%%%%%%%%%%%%%%%%%%%%%%%%%%%%%%%%%%%%%%%%%%%%%%%%%%%%%%%%%%%%%%%%%%%%%%%%%%%

Consider a scattering volume $V$, containing $N$ scattering centers with shape functions $S^i(\vect{r})$, positions $\vect{R}^i$ and scattering length density $\rho_s$ (relative to the ambient material).

In the Born approximation ($\widetilde{T}\simeq\widetilde{V}$), the scattering amplitude is
\begin{align*}
  f(\theta, \phi) & = -8\pi^3 \braket{\vect{k_f}}{\rho_s(\vect{r})}{\vect{k_i}} \nonumber \\
  & = -\int d^3\vect{r} e^{i\vect{q}\cdot\vect{r}} \rho_s(\vect{r}) \; ,
\end{align*}
where $\vect{q}\equiv \vect{k}_i - \vect{k}_f$ denotes the wavevector transfer and
\begin{equation*}
  \rho_s(\vect{r}) = \frac{k_0^2}{4\pi}(1-n^2(\vect{r})) \; .
\end{equation*}

The differential cross-section (per scattering center) is then given by:
\begin{equation*}
  \frac{d\sigma}{d\Omega}(\vect{q}) = \frac{1}{N}\left\lvert \int_V \rho_s(\vect{r}) e^{i\vect{q}\cdot\vect{r}} d^3\vect{r} \right\rvert ^2 \; .
\end{equation*}

Following the initial assumptions. the scattering length density can be written as:
\begin{equation*}
\rho_s(\vect{r}) = \sum_i \rho_{s,i} S^i(\vect{r}) \otimes \delta (\vect{r}-\vect{R}^i) \; ,
\end{equation*}
with $\rho_{s,i}$ the scattering length density of particle $i$. The cross-section then becomes:
\begin{align*}
  N\frac{d\sigma}{d\Omega}(\vect{q}) & = \left\lvert \sum_i F^i(\vect{q}) \exp (i\vect{q}\cdot\vect{R}^i) \right\rvert ^2  \\
  & = \left\lbrace \sum_i \left\lvert F^i(\vect{q}) \right\rvert ^2 + \sum_{i\neq j} F^i(\vect{q}) F^{j*}(\vect{q}) \exp \left[i\vect{q}\cdot (\vect{R}^i-\vect{R}^j)\right] \right\rbrace \; .
\end{align*}
In the last expression, the formfactors $F^i(\vect{q})$ are the Fourier transforms of the shape functions, including their scattering length densities:
\begin{equation*}
  F^i(\vect{q}) \equiv \int d^3\vect{r} \rho_{s,i} S^i(\vect{r}) \exp (i\vect{q}\cdot\vect{r} ) \; .
\end{equation*}

Since in most real conditions only the statistical properties of the particles are known, one can consider the expectation value of this cross-section. Assuming that the particles' shapes are determined by their class $\alpha$, with abundance ratio $p_\alpha \equiv N_\alpha / N$, and defining the particle density $\rho_V \equiv N/V$, the expectation value becomes:
\begin{align*}
  \left\langle \frac{d\sigma}{d\Omega}(\vect{q}) \right\rangle  & = \sum_\alpha p_\alpha \left\lvert F_\alpha(\vect{q})\right\rvert ^2 + \frac{\rho_V}{V}\sum_{\alpha,\beta} p_\alpha p_\beta F_\alpha (\vect{q})F_\beta^*(\vect{q})  \\
  & \times \iint_V d^3\vect{R}_\alpha d^3\vect{R}_\beta \ppcf{\alpha}{\beta}{R} \exp \left[ i\vect{q}\cdot (\vect{R}_\alpha - \vect{R}_\beta ) \right] \; .
\end{align*}

In this equation, the factor $\ppcf{\alpha}{\beta}{R}$ is called the \emph{partial pair correlation function} and it represents a normalized probability of finding particles of type $\alpha$ and $\beta$ in positions $\vect{R}_\alpha$ and $\vect{R}_\beta$ respectively. More precisely, the probability density for finding a particle $\alpha$ at position $\vect{R}_\alpha$ and another one of type $\beta$ at $\vect{R}_\beta$ is given by:
\begin{equation*}
  \mathcal{P}(\alpha, \vect{R}_\alpha ; \beta , \vect{R}_\beta ) \equiv \rho_V^2 p_\alpha p_\beta \ppcf{\alpha}{\beta}{R} \; .
\end{equation*}


%%%%%%%%%%%%%%%%%%%%%%%%%%%%%%%%%%%%%%%%%%%%%%%%%%%%%%%%%%%%%%%%%%%%%%%%%%%%%%%%
\section{Distorted Wave Born Approximation} 
%%%%%%%%%%%%%%%%%%%%%%%%%%%%%%%%%%%%%%%%%%%%%%%%%%%%%%%%%%%%%%%%%%%%%%%%%%%%%%%%
\index{Distorted wave Born approximation}
\index{DWBA|see {Distorted wave Born approximation}}

In this section, one proceeds along similar lines as in the formal treatment of section \ref{sec:formal}. This time however, the full Hamiltonian is written as $H_2 = H_1 + V_2 = H_0 +V_1 + V_2$, where $H_0$ will again refer to the free Hamiltonian. In the distorted wave Born approximation (DWBA), one performs a perturbative expansion around the solutions of the Hamiltonian $H_1$, which are assumed to be known:
\begin{align*}
  H_1\Psi^\pm_{1\alpha} &= E_\alpha\Psi^\pm_{1\alpha} \\
  \Psi^\pm_{1\alpha} &= \Psi_{0\alpha} + G^\pm_1 V_1 \Psi_{0\alpha} \; ,
\end{align*}
where the Green operators are defined to be:
\begin{align*}
  G^\pm_1 &\equiv (E-H_1\pm i\epsilon) ^{-1} \nonumber \\
  G^\pm_2 &\equiv (E-H_2\pm i\epsilon) ^{-1} \; .
\end{align*}

The $T$-matrix element for scattering between the asymptotic states $\Psi_{0\alpha}$ and $\Psi_{0\beta}$ (note that these asymptotic states refer to the free Hamiltonian $H_0$), is:
\begin{align*}
  T^+_{\alpha\beta} &= \braket{\Psi_{0\beta}}{V_1+V_2}{\Psi^+_\alpha} \nonumber \\
  & = \braket{\Psi_{0\beta}}{V_1+V_2}{\Psi_{0\alpha} + G^+_1V_1\Psi_{0\alpha} + G^+_2V_2\Psi^+_{1\alpha}} \nonumber \\
  & = \braket{\Psi_{0\beta}}{V_1}{\Psi^+_{1\alpha}} + \braket{\Psi_{0\beta}}{(V_1G^+_1 + 1)V_2}{\Psi^+_\alpha} \\
  & = \braket{\Psi_{0\beta}}{V_1}{\Psi^+_{1\alpha}} + \braket{\Psi^-_{1\beta}}{V_2}{\Psi^+_\alpha} \nonumber \\
  & = \braket{\Psi_{0\beta}}{V_1}{\Psi^+_{1\alpha}} + \braket{\Psi^-_{1\beta}}{T_2}{\Psi^+_{1\alpha}} \; ,
\end{align*}
with $T_2 = V_2 + V_2G^+_2V_2$. By approximating this last term using $T_2 \simeq V_2$, one arrives at the distorted wave Born approximation:
\begin{equation}
  \label{eq:tdwba}
   T^+_{\alpha\beta} \simeq \braket{\Psi_{0\beta}}{V_1}{\Psi^+_{1\alpha}} + \braket{\Psi^-_{1\beta}}{V_2}{\Psi^+_{1\alpha}} \; .
\end{equation}


%===============================================================================
\subsection{Multilayer systems}
%===============================================================================

In multilayer systems, the first term of \refeq{tdwba} denotes the specular part of the reflection, while the second term is responsible for the off-specular scattering. This off-specular part is caused by deviations from the perfectly smooth layered system, as e.g. interface roughnesses or included nanoparticles. In here only the case of nanoparticles will be treated.

In the conventions where $H=-\Delta + V$, the potential splits into two parts $V_1$ and $V_2$, where only the second part is treated as a perturbation:
\begin{align*}
  V_1 & = k_0^2\left( 1-n_0^2(\vect{r})\right)  \\
  V_2 & = \sum_i k_0^2\left( n_0^2(\vect{R}^i) - n_i^2 \right) S^i(\vect{r}) \otimes \delta(\vect{r}-\vect{R}^i) \; ,
\end{align*}
where $n_0(\vect{r})$ denotes the refractive index of the unperturbed system (which, in case of a multilayer system, will only depend on its $z$-coordinate) and $n_i$ is the refractive index of the nanoparticle with shape function $S^i$ and position $\vect{R}^i$.

For nanoparticles in a specific layer $j$, i.e. $V_2\neq0$ only in layer $j$, one only needs the unperturbed solutions in layer $j$:
\begin{align*}
  \braketnoop{\vect{r}}{\Psi^+_{1k_i}} &= (2\pi)^{-3/2}\left[ R_j(\vect{k}_i) e^{i \vect{k}_{j,R}(\vect{k}_i)\cdot\vect{r}} + T_j(\vect{k}_i) e^{i \vect{k}_{j,T}(\vect{k}_i)\cdot\vect{r}} \right] \\
  \braketnoop{\Psi^-_{1k_f}}{\vect{r}} &= (2\pi)^{-3/2}\left[ R_j(-\vect{k}_f) e^{i \vect{k}_{j,R}(-\vect{k}_f)\cdot\vect{r}} + T_j(-\vect{k}_f) e^{i \vect{k}_{j,T}(-\vect{k}_f)\cdot\vectr} \right] \; .
\end{align*}

The off-specular contribution to the scattering amplitude then becomes:
\begin{align*}
  f(\theta, \phi) &= -\int d^3\vectr \frac{V_2(\vectr)}{4\pi} \biggl[ T_iT_fe^{i(\vectk_{j,i}-\vectk_{j,f})\cdot\vectr} + R_iT_fe^{i(\vectkt_{j,i}-\vectk_{j,f})\cdot\vectr} \\
   & + T_iR_fe^{i(\vectk_{j,i}-\vectkt_{j,f})\cdot\vectr} + R_iR_fe^{i(\vectkt_{j,i}-\vectkt_{j,f})\cdot\vectr} \biggr] \; ,
\end{align*}
where the following shorthand notations were used:
\begin{align*}
  T_i &\equiv  T_j(\vect{k}_i) & R_i &\equiv  R_j(\vect{k}_i)  \\
  T_f &\equiv  T_j(-\vect{k}_f) & R_f &\equiv  R_j(-\vect{k}_f) \\
  \vectk_{j,i} &\equiv \vectk_{j,T}(\vectk_i) & \vectkt_{j,i} &\equiv \vectk_{j,R}(\vectk_i)  \\
  \vectk_{j,f} &\equiv -\vectk_{j,T}(-\vectk_f) & \vectkt_{j,f} &\equiv -\vectk_{j,R}(-\vectk_f) \; .
\end{align*}

From this expression, one sees that the scattering amplitude consists of a weighted sum of Fourier transforms of the potential $V_2$. Using
\begin{equation*}
  V_2(\vectr) = \sum_i 4\pi \rho_{s,rel,i} S^i(\vectr) \otimes \delta(\vectr - \vect{R}^i) \; ,
\end{equation*}
with $\rho_{s,rel,i}\equiv  k_0^2\left( n_0^2(\vect{R}^i) - n_i^2 \right)/4\pi$, the scattering amplitude becomes
\begin{equation*}
  f(\theta, \phi) = -\sum_i  \rho_{s,rel,i} \curlf^i_{\text{DWBA}}(\vectk_{j,i},\vectk_{j,f},\vect{R}^i_z)e^{i(\vectk_{j,i\parallel}-\vectk_{j,f\parallel})\cdot \vect{R}^{i\parallel} } \; ,
\end{equation*}
with
\begin{align*}
  \curlf^i_\text{DWBA}(\vectk_i,\vectk_f,R_z) & \equiv T_iT_fF^i(\vectk_i-\vectk_f)e^{i(k_{iz}-k_{fz})R_z} + R_iT_fF^i(\vectkt_i-\vectk_f)e^{i(-k_{iz}-k_{fz})R_z} \\
  & + T_iR_fF^i(\vectk_i-\vectkt_f)e^{i(k_{iz}+k_{fz})R_z} + R_iR_fF^i(\vectkt_i-\vectkt_f)e^{i(-k_{iz}+k_{fz})R_z} \; ,
\end{align*}

With this last expression, the same techniques as demonstrated in section \ref{sec:ba} can be applied, leading to the following expression for the expectation value of the scattering cross-section:
\begin{align*}
  & \left\langle \frac{d\sigma}{d\Omega}(\vectk_i,\vectk_f) \right\rangle_{\text{Off-specular}}  \\
  & = \sum_\alpha p_\alpha \left\lvert \curlf_\alpha(\vectk_{j,i},\vectk_{j,f}, R_{\alpha,z})\right\rvert ^2 + \frac{\rho_S}{S}\sum_{\alpha,\beta} p_\alpha p_\beta \curlf_\alpha (\vectk_{j,i},\vectk_{j,f}, R_{\alpha,z})\curlf_\beta^*(\vectk_{j,i},\vectk_{j,f}, R_{\beta,z}) \\
  & \times \iint_S d^2\vect{R}_\alpha^\parallel d^2\vect{R}_\beta^\parallel \ppcf{\alpha}{\beta}{R^\parallel} \exp \left[ i\vect{q}_{j\parallel}\cdot (\vect{R}_\alpha^\parallel - \vect{R}_\beta^\parallel ) \right] \; .
\end{align*}

The main differences with respect to the cross-section in the Born approximation are:
\begin{enumerate}
  \item The particle form factor now consists of a more complex expression and now depends on both incoming and outgoing wavevectors and also on the $z$-coordinate of the particle;
  \item Since the $z$-coordinate of the particles is implicitly included in its formfactor, the position integrals only run over $x$- and $y$-coordinates and the volume and density gets replaced with the surface area and surface density respectively.
\end{enumerate}


