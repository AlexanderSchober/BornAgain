%%%%%%%%%%%%%%%%%%%%%%%%%%%%%%%%%%%%%%%%%%%%%%%%%%%%%%%%%%%%%%%%%%%%%%%%%%%%%%%%
%%
%%   BornAgain User Manual
%%
%%   homepage:   http://www.bornagainproject.org
%%
%%   copyright:  Forschungszentrum Jülich GmbH 2015
%%
%%   license:    Creative Commons CC-BY-SA
%%   
%%   authors:    Scientific Computing Group at MLZ Garching
%%               C. Durniak, M. Ganeva, G. Pospelov, W. Van Herck, J. Wuttke
%%
%%%%%%%%%%%%%%%%%%%%%%%%%%%%%%%%%%%%%%%%%%%%%%%%%%%%%%%%%%%%%%%%%%%%%%%%%%%%%%%%


\cleardoublepage
\mychapter{0}{Introduction}

\BornAgain\ is a free and open-source software package
to simulate and fit small-angle
scattering at grazing incidence (GISAS). 
It supports analysis of both X-ray (GISAXS) and neutron (GISANS) data.
Its name, \BornAgain,
indicates the central role of the distorted-wave Born
approximation (DWBA) in the physical description of the
scattering process.
\index{Distorted-wave Born approximation}
The software provides a generic framework for modeling multilayer samples with smooth or
rough interfaces and with various types of embedded nano\-particles.

\BornAgain\ almost completely reproduces the functionality
of the widely used program \IsGISAXS\
\index{IsGISAXS@\IsGISAXS}
\index{Lazzari, R\'emi}
by R\'emi Lazzari \cite{Laz02}.
\BornAgain\ goes beyond \IsGISAXS\ by
supporting an unrestricted number of layers and particles, 
diffuse reflection from rough layer interfaces,
particles with inner structures, neutron polarization and magnetic scattering.
Adhering to a strict object-oriented design,
\BornAgain\ provides a solid base for future extensions
in response to specific user needs.

\BornAgain\ is a platform-independent software,
with active support for
Linux,\index{Linux}
MacOS\index{MacOS}
and  Microsoft Windows.\index{Windows|see {Microsoft Windows}}
\index{Microsoft Windows}
It is a free and open source software provided under the terms
of the GNU General Public License (GPL, version 3 or higher).
This documentation is released under the Creative Commons license CC-BY-SA.
When \BornAgain\ is used in preparing scientific papers,
please cite software and manual as follows:
\index{Citation}
\begin{quote}
C.~Durniak, M.~Ganeva, G.~Pospelov, W.~Van Herck, J.~Wuttke (2015),\newline
BornAgain --- Software for simulating and fitting
X-ray and neutron small-angle scattering at grazing incidence,
version \UserManualVersionNumber,\newline
\url{http://www.bornagainproject.org}
\end{quote}

This User Manual is complementary to the online documentation
at \url{http://www.bornagainproject.org}.
It does not duplicate information that is more conveniently read online.
Therefore, Sect.~\ref{sec:online} just contains a few pointers to the web site.
The remainder of this User Manual mostly contains background
on the sample models and on the scattering theory implemented in \BornAgain,
and some documentation of the corresponding \Python\ functions.
%Sect.~\ref{sec:Simulation} describes
%the general methodology of a simulation with \BornAgain,
%and gives detailed usage examples.
%Sect.~\ref{sec:ScatteringCrosssection} explains
%which sample structures are supported in \BornAgain,
%and which physical approximations are used.
%Fitting is explained in Sect.~\ref{sec:Fitting}.
%More theoretical background is given in Appendix~\ref{app:theory}.
%Implemented particle formfactors are specified in Appendix~\ref{app:ff}.

\BareWarning{Software and documentation are work in progress.
We cannot guarantee that they are accurate and correct.
Anyway, it is entirely in the responsibility of users
to ensure that their data interpretation is physically meaningful.
If in doubt, please contact us.}

\index{Bug reports}
We are grateful for all kind of feedback:
criticism, praise, bug reports, feature requests or contributed modules.
If questions go beyond normal user support,
we will be glad to discuss a scientific collaboration.
