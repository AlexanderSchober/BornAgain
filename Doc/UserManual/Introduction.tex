\newpage
\mychapter{0}{Introduction}

%\BornAgain\ is a  software to simulate and fit neutron and X-ray
%scattering at grazing incidence. It is a multi–platform open–source project that aims at %supporting scientists in the analysis and fitting
%of their GISAS data, both for synchrotron (GISAXS) and neutron (GISANS) facilities. 

\BornAgain\ is a free open-source software package to simulate and fit small-angle
scattering at grazing incidence (GISAS). 
It supports analysis of both  X-ray (GISAXS) and neutron (GISANS) data.
Its name, \BornAgain, indicates the central role of the distorted-wave Born
approximation (DWBA) in the physical description of the
scattering process. The software provides a generic framework for modeling multilayer samples with smooth or
rough interfaces and with various types of embedded nano\-particles.

\BornAgain\ almost completely reproduces the functionality
of the widely used program \IsGISAXS\
by R. Lazzari \cite{Laz02}.
%\BornAgain\ is intended  to reproduce the functionality
%of the widely used program \IsGISAXS\
%by R. Lazzari \cite{Laz02}.

\BornAgain\ goes beyond \IsGISAXS\ by
supporting an unrestricted number of layers and particles, 
diffuse reflection from rough layer interfaces,
particles with inner structures, neutron polarization and magnetic scattering.
Adhering to a strict object-oriented design,
\BornAgain\ provides a solid base for future extensions
in response to specific user needs.

\BornAgain\ is a platform-independent software,
with active support for Linux, MacOS and 
Microsoft Windows. 
It is a free and open source software provided under the terms
of the GNU General Public License (GPL).
This documentation is released under the Creative Commons license CC-BY-SA.

The authors will be grateful for all kind of
feedback: criticism, praise, bug reports, feature requests
or contributed modules.
When \BornAgain\ is used in preparing scientific papers,
please cite software and manual as follows: 
\begin{quote}
C. Durniak, M. Ganeva, G. Pospelov, W. Van Herck, J. Wuttke (2015),\newline
BornAgain --- Software for simulating and fitting
X-ray and neutron small-angle scattering at grazing incidence,
version \UserManualVersionNumber,\newline
\url{http://www.bornagainproject.org}
\end{quote}

%For details about the theory (DWBA,\ldots), please refer to IsGISAXS manual \cite{IsGISAXSManual}.


%This user guide starts with a brief description of the steps necessary for compiling 
%the source code and running
%the simulation in \SecRef{QuickStart}. More detailed overview of software architecture and
%installation procedure are given in \SecRef{SoftwareArchitecture} and %\SecRef{Installation}.
%General methodology of simulation with \BornAgain\ and detailed usage examples are given
%in Chapter~\ref{ExamplesChapter}.%\SecRef{Examples}.
%Fitting tools provided by the framework are presented in \SecRef{Fitting}.

This user guide starts with a brief description of the steps necessary for 
installing the software and running a simulation 
on Windows, Mac OS X and Unix platforms in \SecRef{QuickStart}.
%A more detailed description of the installation procedure is given in \SecRef{Installation}.
%\SecRef{SoftwareArchitecture} provides brief overview of software architecture,
%while general methodology of simulation with \BornAgain\ and detailed usage examples are %given in \SecRef{Simulation}.
%Fitting toolkit provided by the framework are presented in \SecRef{Fitting}.
The general methodology of a simulation with \BornAgain\ and detailed simulation usage examples are given in \SecRef{Simulation}. 
\SecRef{ScatteringCrosssection} describes the way the scattering cross--section is calculated.
The fitting toolkit, provided by the framework, is presented in \SecRef{Fitting}.


%\vspace*{2mm}
%
%\colorbox{Lightgray}{\parbox{0.95\linewidth}
%{
%\noindent \underline{Icons used in this manual:}
%\begin{itemize}
%\item[] \smallpencil: this sign highlights further remarks.
%%\item[] {\huge\ding{45}} - \smallpencil \textbf{\smallpencil}: this sign highlights further references.
%\item[] {\huge\danger}: this sign highlights essential points.
%\end{itemize}
%}
%}
