\section{Basic Python fitting example.} \SecLabel{FittingExamples}

In this section we are going to go through a complete example of
fitting using \BornAgain. Each of the steps will be associated with a
detailed piece of code written in Python. 
The complete listing of
the script is given at the end (see Listing~\ref{PythonFittingExampleScript}).
Script itself can be found at
\begin{lstlisting}[language=shell, style=commandline]
./Examples/python/fitting/ex002_FitCylindersAndPrisms/FitCylindersAndPrisms.py
\end{lstlisting}

\noindent
The example uses same sample geometry as in \SecRef{Example1Python}.
It represents cylindrical and
prismatic particles in equal proportion, in an air layer, deposited on a substrate layer, with no interference
between the particles. We consider following parameters to be unkown
\begin{itemize}
\item the radius of cylinders
\item the height of cylinders
\item half side length of the prisms' triangular basis
\item the height of prisms
\end{itemize}

Our reference data are a ``noisy'' two-dimensional intensity
map obtained from the simulation of the same geometry with a fixed
value of $5\,{\rm nm}$ for all four of these parameters. 
Then we run our fitting using default minimizer settings
starting with a cylinder's height
of $4\,{\rm nm}$, a cylinder's radius of $6\,{\rm nm}$, 
a prism's half side of $6\,{\rm nm}$ and a length equal to $4\,{\rm nm}$.
As a result, fitting procedure is able to restore correct value of $5\,{\rm nm}$
for all parameters.


%%%%%%%%%%%%%%%%%%%%%%%%%%%%%%%%%%%%%%%%%%%%%%%%%%%%%%%%%%%%%%%%%%%%%%%%%%%%%%%
\subsubsection*{Importing Python libraries}
\begin{lstlisting}[language=python, style=eclipseboxed]
from libBornAgainCore import *
from libBornAgainFit import *
\end{lstlisting}
We start from importing two \BornAgain\ libraries required to create sample description
and to run the fitting.


%%%%%%%%%%%%%%%%%%%%%%%%%%%%%%%%%%%%%%%%%%%%%%%%%%%%%%%%%%%%%%%%%%%%%%%%%%%%%%%
\subsubsection*{Building the sample}
\begin{lstlisting}[language=python, style=eclipseboxed, firstnumber=5]
def get_sample(): @\label{script2::get_sample}@
    """
    Build the sample representing cylinders and pyramids on top of substrate without interference.
    """
    # defining materials
    m_air = MaterialManager.getHomogeneousMaterial("Air", 0.0, 0.0)
    m_substrate = MaterialManager.getHomogeneousMaterial("Substrate", 6e-6, 2e-8)
    m_particle = MaterialManager.getHomogeneousMaterial("Particle", 6e-4, 2e-8)

    # collection of particles
    cylinder_ff = FormFactorCylinder(1.0*nanometer, 1.0*nanometer)
    cylinder = Particle(m_particle, cylinder_ff)
    prism_ff = FormFactorPrism3(1.0*nanometer, 1.0*nanometer)
    prism = Particle(m_particle, prism_ff)
    particle_decoration = ParticleDecoration()
    particle_decoration.addParticle(cylinder, 0.0, 0.5)
    particle_decoration.addParticle(prism, 0.0, 0.5)
    interference = InterferenceFunctionNone()
    particle_decoration.addInterferenceFunction(interference)

    # air layer with particles and substrate form multi layer
    air_layer = Layer(m_air)
    air_layer.setDecoration(particle_decoration)
    substrate_layer = Layer(m_substrate, 0)
    multi_layer = MultiLayer()
    multi_layer.addLayer(air_layer)
    multi_layer.addLayer(substrate_layer)
    return multi_layer
\end{lstlisting}
Function starting at the line ~\ref{script2::get_sample} creates multilayered sample
with cylinders and prisms using arbitrary $1\,{\rm nm}$ value for all size's of particles.
The details about the generation of this multilayered sample are given in \SecRef{Example1Python}.


%%%%%%%%%%%%%%%%%%%%%%%%%%%%%%%%%%%%%%%%%%%%%%%%%%%%%%%%%%%%%%%%%%%%%%%%%%%%%%%
\subsubsection*{Creating the simulation.}
\begin{lstlisting}[language=python, style=eclipseboxed, firstnumber=35]
def get_simulation(): @\label{script2::get_simulation}@
    """
    Create GISAXS simulation with beam and detector defined
    """
    simulation = Simulation()
    simulation.setDetectorParameters(100, -1.0*degree, 1.0*degree, 100, 0.0*degree, 2.0*degree, True)
    simulation.setBeamParameters(1.0*angstrom, 0.2*degree, 0.0*degree)
    return simulation
\end{lstlisting}
Function starting at the line ~\ref{script2::get_simulation} creates
simulation object with beam and detector parameters defined.



%%%%%%%%%%%%%%%%%%%%%%%%%%%%%%%%%%%%%%%%%%%%%%%%%%%%%%%%%%%%%%%%%%%%%%%%%%%%%%%
\subsubsection*{Preparing the fitting pair.}
\begin{lstlisting}[language=python, style=eclipseboxed, firstnumber=45]
def run_fitting(): @\label{script2::run_fitting}@
    """
    run fitting
    """
    sample = get_sample() @\label{script2::setup_simulation1}@
    simulation = get_simulation()
    simulation.setSample(sample) @\label{script2::setup_simulation2}@

    real_data = OutputDataIOFactory.readIntensityData('refdata_fitcylinderprisms.txt') @\label{script2::real_data}@
\end{lstlisting}
Lines ~\ref{script2::setup_simulation1}-~\ref{script2::setup_simulation2} generate
sample and simulation description and assign the sample to the simulation.
Our reference data are contained in the file \Code{'refdata\_fitcylinderprisms.txt'}.
 In our case this reference had been generated by adding noise
on the scattered intensity from a numerical sample with a fixed length of 5~nm of the four fitting
parameters (\textit{i.e.} the dimensions of the cylinders and prisms).
Line ~\ref{script2::real_data} creates real data object by loading ASCII data from the file.


%%%%%%%%%%%%%%%%%%%%%%%%%%%%%%%%%%%%%%%%%%%%%%%%%%%%%%%%%%%%%%%%%%%%%%%%%%%%%%%
\subsubsection*{Setting up \rm\bf{FitSuite}.}
\begin{lstlisting}[language=python, style=eclipseboxed, firstnumber=55]
    fit_suite = FitSuite() @\label{script2::fitsuite1}@
    fit_suite.addSimulationAndRealData(simulation, real_data) @\label{script2::fitsuite2}@
    fit_suite.initPrint(10) @\label{script2::fitsuite3}@
\end{lstlisting}
Line ~\ref{script2::fitsuite1} creates a \Code{FitSuite} object which provides
the main interface to the minimization kernel of \BornAgain\ . 
Line ~\ref{script2::fitsuite2} submits simulation description and real data pair to the 
subsequent fitting. Line ~\ref{script2::fitsuite3} set up \Code{FitSuite} to print on
the screen the information about fit progress every $10^{{\rm th}}$ iteration.
\begin{lstlisting}[language=python, style=eclipseboxed, firstnumber=60]
    fit_suite.addFitParameter("*FormFactorCylinder/height", 4.*nanometer, 0.01*nanometer, AttLimits.lowerLimited(0.01)) @\label{script2::fitpars1}@
    fit_suite.addFitParameter("*FormFactorCylinder/radius", 6.*nanometer, 0.01*nanometer, AttLimits.lowerLimited(0.01))
    fit_suite.addFitParameter("*FormFactorPrism3/height", 4.*nanometer, 0.01*nanometer, AttLimits.lowerLimited(0.01))
    fit_suite.addFitParameter("*FormFactorPrism3/half_side", 6.*nanometer, 0.01*nanometer, AttLimits.lowerLimited(0.01)) @\label{script2::fitpars2}@
\end{lstlisting}
Lines ~\ref{script2::fitpars1}--~\ref{script2::fitpars2} enter the
list of fitting parameters. Here we use the cylinders' height and
radius and the prisms' height and half side length. The syntax of
\Code{addFitParameter} is
\begin{lstlisting}[language=python, style=eclipse,numbers=none]
FitSuite().addFitParameter(<name>, <initial value>, <iteration step>, <limits>)
\end{lstlisting}
where \Code{<name>} is the name of sample pool parameters (see \SecRef{WorkingWithSampleParameters}
) selected
as a fitting parameter. Then we input its initial
value and the iteration step used in the minimization process. Finally
\Code{<limits>} specify the boundaries of the parameter's value. Here
the cylinder's length and prism half side are initially equal to $4\,{\rm nm}$,
whereas the cylinder's radius and the prism length are equal to $6\,{\rm nm}$ before the minimization. The
iteration step is equal to $0.01\,{\rm nm}$ and the boundaries are imposed only
on the lower one of $0.01\,{\rm nm}$.


%%%%%%%%%%%%%%%%%%%%%%%%%%%%%%%%%%%%%%%%%%%%%%%%%%%%%%%%%%%%%%%%%%%%%%%%%%%%%%%
\subsubsection*{Running the fit and accessing results}
\begin{lstlisting}[language=python, style=eclipseboxed, firstnumber=66]
    fit_suite.runFit() @\label{script2::fitresults1}@

    print "Fitting completed."
    fit_suite.printResults()@\label{script2::fitresults2}@
    print "chi2:", fit_suite.getMinimizer().getMinValue() 
    fitpars = fit_suite.getFitParameters()
    for i in range(0, fitpars.size()):
        print fitpars[i].getName(), fitpars[i].getValue(), fitpars[i].getError() @\label{script2::fitresults3}@
\end{lstlisting}
Line ~\ref{script2::fitresults1} shows the command to start the fitting process.
During the fitting the progress will be displayed on the screen.
Lines ~\ref{script2::fitresults2}--~\ref{script2::fitresults3} shows different ways of
accessing to fit results.


More details about fitting, access to its results and visualization of fit progress using matplotlib libraries can be learned from detailed example
\begin{lstlisting}[language=shell, style=commandline]
./Examples/python/fitting/ex002_FitCylindersAndPrisms/FitCylindersAndPrisms_detailed.py
\end{lstlisting}

