\section{Basic Python fitting example} \SecLabel{FittingExamples}

In this section we are going to go through a complete example of
fitting using \BornAgain. Each  step will be associated with a
detailed piece of code written in \Python. 
The complete listing of
the script is given in Appendix (see Listing~\ref{PythonFittingExampleScript}).
The script can also be found at
\begin{lstlisting}[language=shell, style=commandline]
./Examples/python/fitting/ex002_FitCylindersAndPrisms/FitCylindersAndPrisms.py
\end{lstlisting}

\noindent
This example uses the same sample geometry as in \SecRef{Example1Python}.
Cylindrical and
prismatic particles in equal proportion are deposited on a substrate layer, with no interference
between the particles. We consider the following parameters to be unkown
\begin{itemize}
\item the radius of cylinders,
\item the height of cylinders,
\item the length of the prisms' triangular basis,
\item the height of prisms.
\end{itemize}

Our reference data are a ``noisy'' two-dimensional intensity
map obtained from the simulation of the same geometry with a fixed
value of $5\,{\rm nm}$ for the height and radius of cylinders and for the
height of prisms which have a 10-nanometer-long side length. 
Then we run our fitting using default minimizer settings
starting with a cylinder's height
of $4\,{\rm nm}$, a cylinder's radius of $6\,{\rm nm}$, 
a prism's half side of $6\,{\rm nm}$ and a height equal to $4\,{\rm nm}$.
As a result, the fitting procedure is able to find the correct value of $5\,{\rm nm}$
for all four parameters.


%%%%%%%%%%%%%%%%%%%%%%%%%%%%%%%%%%%%%%%%%%%%%%%%%%%%%%%%%%%%%%%%%%%%%%%%%%%%%%%
\subsubsection*{Importing Python libraries}
\begin{lstlisting}[language=python, style=eclipseboxed]
from libBornAgainCore import *
from libBornAgainFit import *
\end{lstlisting}
We start from importing two \BornAgain\ libraries required to create
the sample description
and to run the fitting.


%%%%%%%%%%%%%%%%%%%%%%%%%%%%%%%%%%%%%%%%%%%%%%%%%%%%%%%%%%%%%%%%%%%%%%%%%%%%%%%
\subsubsection*{Building the sample}
\begin{lstlisting}[language=python, style=eclipseboxed, firstnumber=5]
def get_sample(): @\label{script2::get_sample}@
    """
    Build the sample representing cylinders and pyramids on top of substrate without interference.
    """
    # defining materials
    m_air = HomogeneousMaterial("Air", 0.0, 0.0)
    m_substrate = HomogeneousMaterial("Substrate", 6e-6, 2e-8)
    m_particle = HomogeneousMaterial("Particle", 6e-4, 2e-8)

    # collection of particles
    cylinder_ff = FormFactorCylinder(1.0*nanometer, 1.0*nanometer)
    cylinder = Particle(m_particle, cylinder_ff)
    prism_ff = FormFactorPrism3(2.0*nanometer, 1.0*nanometer)
    prism = Particle(m_particle, prism_ff)
    particle_layout = ParticleLayout()
    particle_layout.addParticle(cylinder, 0.0, 0.5)
    particle_layout.addParticle(prism, 0.0, 0.5)
    interference = InterferenceFunctionNone()
    particle_layout.addInterferenceFunction(interference)

    # air layer with particles and substrate form multi layer
    air_layer = Layer(m_air)
    air_layer.setLayout(particle_layout)
    substrate_layer = Layer(m_substrate)
    multi_layer = MultiLayer()
    multi_layer.addLayer(air_layer)
    multi_layer.addLayer(substrate_layer)
    return multi_layer
\end{lstlisting}
The function starting at line~\ref{script2::get_sample} creates a multilayered sample
with cylinders and prisms using arbitrary $1\,{\rm nm}$ value for all size's of particles.
The details about the generation of this multilayered sample are given in \SecRef{Example1Python}.


%%%%%%%%%%%%%%%%%%%%%%%%%%%%%%%%%%%%%%%%%%%%%%%%%%%%%%%%%%%%%%%%%%%%%%%%%%%%%%%
\subsubsection*{Creating the simulation}
\begin{lstlisting}[language=python, style=eclipseboxed, firstnumber=35]
def get_simulation(): @\label{script2::get_simulation}@
    """
    Create GISAXS simulation with beam and detector defined
    """
    simulation = Simulation()
    simulation.setDetectorParameters(100, -1.0*degree, 1.0*degree, 100, 0.0*degree, 2.0*degree)
    simulation.setBeamParameters(1.0*angstrom, 0.2*degree, 0.0*degree)
    return simulation
\end{lstlisting}
The function starting at line~\ref{script2::get_simulation} creates
the simulation object with the definition of the beam and detector parameters.



%%%%%%%%%%%%%%%%%%%%%%%%%%%%%%%%%%%%%%%%%%%%%%%%%%%%%%%%%%%%%%%%%%%%%%%%%%%%%%%
\subsubsection*{Preparing the fitting pair}
\begin{lstlisting}[language=python, style=eclipseboxed, firstnumber=45]
def run_fitting(): @\label{script2::run_fitting}@
    """
    run fitting
    """
    sample = get_sample() @\label{script2::setup_simulation1}@
    simulation = get_simulation()
    simulation.setSample(sample) @\label{script2::setup_simulation2}@

    real_data = OutputDataIOFactory.readIntensityData('refdata_fitcylinderprisms.txt') @\label{script2::real_data}@
\end{lstlisting}
Lines
~\ref{script2::setup_simulation1}-~\ref{script2::setup_simulation2}
generate the 
sample and simulation description and assign the sample to the simulation.
Our reference data are contained in the file \Code{'refdata\_fitcylinderprisms.txt'}.
 This reference had been generated by adding noise
on the scattered intensity from a numerical sample with a fixed length of 5~nm for the four fitting
parameters (\textit{i.e.} the dimensions of the cylinders and prisms).
Line ~\ref{script2::real_data} creates the real data object by loading
the ASCII data from the input file.


%%%%%%%%%%%%%%%%%%%%%%%%%%%%%%%%%%%%%%%%%%%%%%%%%%%%%%%%%%%%%%%%%%%%%%%%%%%%%%%
\subsubsection*{Setting up \rm\bf{FitSuite}}
\begin{lstlisting}[language=python, style=eclipseboxed, firstnumber=55]
    fit_suite = FitSuite() @\label{script2::fitsuite1}@
    fit_suite.addSimulationAndRealData(simulation, real_data) @\label{script2::fitsuite2}@
    fit_suite.initPrint(10) @\label{script2::fitsuite3}@
\end{lstlisting}
Line ~\ref{script2::fitsuite1} creates a \Code{FitSuite} object which provides
the main interface to the minimization kernel of \BornAgain\ . 
Line ~\ref{script2::fitsuite2} submits simulation description and real data pair to the 
subsequent fitting. Line ~\ref{script2::fitsuite3} sets up \Code{FitSuite} to print on
the screen the information about fit progress once per 10 iterations.
\begin{lstlisting}[language=python, style=eclipseboxed, firstnumber=60]
    fit_suite.addFitParameter("*FormFactorCylinder/height", 4.*nanometer, 0.01*nanometer, AttLimits.lowerLimited(0.01)) @\label{script2::fitpars1}@
    fit_suite.addFitParameter("*FormFactorCylinder/radius", 6.*nanometer, 0.01*nanometer, AttLimits.lowerLimited(0.01))
    fit_suite.addFitParameter("*FormFactorPrism3/height", 4.*nanometer, 0.01*nanometer, AttLimits.lowerLimited(0.01))
    fit_suite.addFitParameter("*FormFactorPrism3/length", 12.*nanometer, 0.02*nanometer, AttLimits.lowerLimited(0.01)) @\label{script2::fitpars2}@
\end{lstlisting}
Lines ~\ref{script2::fitpars1}--~\ref{script2::fitpars2} enter the
list of fitting parameters. Here we use the cylinders' height and
radius and the prisms' height and side length. 
The cylinder's length and prism half side are initially equal to $4\,{\rm nm}$,
whereas the cylinder's radius and the prism half side length are equal to $6\,{\rm nm}$ before the minimization. The
iteration step is equal to $0.01\,{\rm nm}$ and only the lower
boundary is imposed to be equal to $0.01\,{\rm nm}$.


%%%%%%%%%%%%%%%%%%%%%%%%%%%%%%%%%%%%%%%%%%%%%%%%%%%%%%%%%%%%%%%%%%%%%%%%%%%%%%%
\subsubsection*{Running the fit and accessing results}
\begin{lstlisting}[language=python, style=eclipseboxed, firstnumber=66]
    fit_suite.runFit() @\label{script2::fitresults1}@

    print "Fitting completed."
    fit_suite.printResults()@\label{script2::fitresults2}@
    print "chi2:", fit_suite.getMinimizer().getMinValue() 
    fitpars = fit_suite.getFitParameters()
    for i in range(0, fitpars.size()):
        print fitpars[i].getName(), fitpars[i].getValue(), fitpars[i].getError() @\label{script2::fitresults3}@
\end{lstlisting}
Line ~\ref{script2::fitresults1} shows the command to start the fitting process.
During the fitting the progress will be displayed on the screen.
Lines ~\ref{script2::fitresults2}--~\ref{script2::fitresults3} shows different ways of
accessing the fit results.


More details about fitting, access to its results and visualization of
the fit progress using matplotlib libraries can be learned from the
following detailed example
\begin{lstlisting}[language=shell, style=commandline]
./Examples/python/fitting/ex002_FitCylindersAndPrisms/FitCylindersAndPrisms_detailed.py
\end{lstlisting}

